\chapter{Groups}

\section{Definitions}

\begin{definition}[Binary Operation]
    Let \(G\) be a set. The function \(*: G \times G \to G\) is called a \textit{binary operation} on \(G\). Symbolically, let \(a, b, c \in G\), the following expressions are equivalent:
    \[
        *(a, b) = c \iff a * b = c
    \]
\end{definition}

\begin{remark}
    The function \(f: G \to G\) is called a \textit{unary operation} on \(G\). This is also known as a \textit{discrete dynamical system}.
\end{remark}

\begin{nexample}
    Let \(G = \mathbb{Z}^+\). We have the following binary operations:
    \begin{itemize}
        \item Addition, \(+: G \times G \to G\)
        \item Multiplication, \(\times: G \times G \to G\)
    \end{itemize}
    Notice that these have the closure property. Subtraction and divisions, however, are not binary operations on \(\mathbb{Z}^+\) as they do not have the closure property.
\end{nexample}

\begin{definition}[Group]
    Let \(G\) be a set with a binary operation \(*\). \((G, *)\) is a group if the following conditions are satisfied:
    \begin{enumerate}
        \item Associativity: For every \(a, b, c \in G\), we have that
            \[
                (a * b) * c = a * (b * c)
            \]
            Intuitively, it just means that parentheses don't matter.
        \item Identity: There exists element \(e \in G\) such that for every \(a \in G\), we have that
            \[
                a * e = a = e * a
            \]
        \item Inverse: For every \(a \in G\), there is \(b \in G\) such that
            \[
                a * b = b * a = e
            \]
    \end{enumerate}
\end{definition}

\begin{remark}
    In many cases, if the operation \(*\) is clear in the context, we usually write \(a * b = ab\) for some \(a, b \in G\).

    Also, if \(ab = ba\) for all \(a, b \in G\), then \(G\) is called Abelian---similar to a mention in chapter 1.
\end{remark}

\begin{nexample}
    Some of the Abelian groups are \((\mathbb{Z}, +), (\mathbb{Q}, +), (\mathbb{R}, +), (\mathbb{C}, +)\). However, \((\mathbb{Z}, {}\cdot{})\) is not Abelian. In fact, it's not even a group! This is because there is no inverse for \(0 \in \mathbb{Z}\). Even when 0 is removed from \(\mathbb{Z}\), \((\mathbb{Z} \setminus 0, {}\cdot{})\) is still not a group as it still does not contain any inverse.
\end{nexample}

\begin{nexample}
    Here are a quick rapid-fire of a pairing of a set and a binary operation:
    \begin{itemize}
        \item \((\mathbb{Q} \setminus 0, {}\cdot{})\) is a group
        \item \(\left(M_{2 \times 2}(\mathbb{R}), +\right)\) is a group
        \item \(\left(M_{2 \times 2}(\mathbb{R}), {}\cdot{}\right)\) is not a group
        \item \(\left(GL_2(\mathbb{R}) = \left\{A \in M_{2 \times 2}(\mathbb{R}) : \det(A) \neq 0\right\}, {}\cdot{}\right)\) is a group
    \end{itemize}
\end{nexample}

\begin{theorem}[Uniqueness of Identity]
    Let \((G, {}*{})\) be a group. \(G\) has a unique identity.
\end{theorem}

\begin{proof}
    Since \(G\) is a group, there is an identity, say \(e \in G\). Let us assume that there is another identity, say \(e' \in G\). We have now have that
    \[
    \begin{aligned}
        e &= e * e' \\
          &= e' * e \\
          &= e'
    \end{aligned}
    \]
    Therefore, we can conclude that the identity is unique.
\end{proof}
