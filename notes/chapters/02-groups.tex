\chapter{Groups}

\section{Definitions}

\begin{definition}[Binary Operation]
    Let \(G\) be a set. The function \(*: G \times G \to G\) is called a \textit{binary operation} on \(G\). Symbolically, let \(a, b, c \in G\), the following expressions are equivalent:
    \[
        *(a, b) = c \iff a * b = c
    \]
\end{definition}

\begin{remark}
    The function \(f: G \to G\) is called a \textit{unary operation} on \(G\). This is also known as a \textit{discrete dynamical system}.
\end{remark}

\begin{nexample}
    Let \(G = \mathbb{Z}^+\). We have the following binary operations:
    \begin{itemize}
        \item Addition, \(+: G \times G \to G\)
        \item Multiplication, \(\times: G \times G \to G\)
    \end{itemize}
    Notice that these have the closure property. Subtraction and divisions, however, are not binary operations on \(\mathbb{Z}^+\) as they do not have the closure property.
\end{nexample}

\begin{definition}[Group]
    Let \(G\) be a set with a binary operation \(*\). \((G, *)\) is a group if the following conditions are satisfied:
    \begin{enumerate}
        \item Associativity: For every \(a, b, c \in G\), we have that
            \[
                (a * b) * c = a * (b * c)
            \]
            Intuitively, it just means that parentheses don't matter.
        \item Identity: There exists element \(e \in G\) such that for every \(a \in G\), we have that
            \[
                a * e = a = e * a
            \]
        \item Inverse: For every \(a \in G\), there is \(b \in G\) such that
            \[
                a * b = b * a = e
            \]
    \end{enumerate}
\end{definition}

\begin{remark}
    In many cases, if the operation \(*\) is clear in the context, we usually write \(a * b = ab\) for some \(a, b \in G\).
\end{remark}

\begin{remark}
    If \(ab = ba\) for all \(a, b \in G\), then \(G\) is called Abelian---similar to a mention in chapter 1.   
\end{remark}

\begin{nexample}
    Some of the Abelian groups are \((\mathbb{Z}, +), (\mathbb{Q}, +), (\mathbb{R}, +), (\mathbb{C}, +)\). However, \((\mathbb{Z}, {}\cdot{})\) is not Abelian. In fact, it's not even a group! This is because there is no inverse for \(0 \in \mathbb{Z}\). Even when 0 is removed from \(\mathbb{Z}\), \((\mathbb{Z} \setminus 0, {}\cdot{})\) is still not a group as it still does not contain any inverse.
\end{nexample}

\begin{nexample}
    Here are a quick rapid-fire of a pairing of a set and a binary operation:
    \begin{itemize}
        \item \((\mathbb{Q} \setminus 0, {}\cdot{})\) is a group
        \item \(\left(M_{2 \times 2}(\mathbb{R}), +\right)\) is a group
        \item \(\left(M_{2 \times 2}(\mathbb{R}), {}\cdot{}\right)\) is not a group
        \item \(\left(GL_2(\mathbb{R}) = \left\{A \in M_{2 \times 2}(\mathbb{R}) : \det(A) \neq 0\right\}, {}\cdot{}\right)\) is a group
    \end{itemize}
\end{nexample}

\begin{theorem}[Uniqueness of Identity]
    Let \((G, {}*{})\) be a group. \(G\) has a unique identity.
\end{theorem}

\begin{proof}
    Since \(G\) is a group, there is an identity, say \(e \in G\). Let us assume that there is another identity, say \(e' \in G\). We have now have that
    \[
    \begin{aligned}
        e &= e * e' \\
          &= e' * e \\
          &= e'
    \end{aligned}
    \]
    Therefore, we can conclude that the identity is unique.
\end{proof}

\section{Additive and Multiplicative Groups with Modulo}

\begin{definition}[Addition with Modulo \(n\) Group]
    We define a new set as follows:
    \[
        \mathbb{Z}_n = \left\{[0], [1], \ldots, [n-1]\right\}
    \]
    where \([k] = \left\{r \in \mathbb{Z} : n \mid k-r \text{ or } r \equiv k \mod n\right\}\) or its equivalence class under modulo \(n\). Intuitively, the equivalence class of \(k\) with modulo \(n\) or \([k]\) can be thought of as the set of integers \(a\) such that \(a \mod n = k\).

    We define their addition to be the set of all possible sums of the elements of each equivalence class. Notice that
    \begin{enumerate}
        \item \(\mathbb{Z}_n\) has associativity: \([i] + [j] &= [i + j] = [j + i] = [j] + [i]\), and
        \item \(Z_n\) has an identity and every element in it has an inverse: \([i] - [i] = [0]\)
    \end{enumerate}
    Thus, we have that \((\mathbb{Z}_n, +)\) is a group.
\end{definition}

\begin{remark}
    Depending on the source, we can denote \(\mathbb{Z}_n = \left\{\overline{0}, \overline{1}, \ldots, \overline{n-1}\right\}\) as well.
\end{remark}

\begin{definition}[Multiplication with Modulo \(n\) Group]
    Define the multiplication with modulo \(n\) group as follows:
    \[
        U(n) = \left\{[i] \in \mathbb{Z}_n : \ \exists \  [j] \in \mathbb{Z}, [i] \cdot [j] = [1]\right\}
    \]
    where the operation is multiplication of these sets. The opration works analogously with the addition with modulo from before.
\end{definition}

Intuitively, \(U(n)\) defines the set of elements of \(\mathbb{Z}_n\) that has a multiplicative inverse.

For a quick-ish computation of how to find which elements show up in \(U(n)\), we can enumerate \(k = 1, 2, \ldots, n-1\) and pick \(k\)'s such that \(\gcd(k, n) = 1\).

\begin{nexample}
    Here are a few examples:
    \[
    \begin{aligned}
        U(2) &= \left\{[1]\right\} \\
        U(3) &= \left\{[1], [2]\right\} \\
        U(4) &= \left\{[1], [3]\right\}
    \end{aligned}
    \]
\end{nexample}

\begin{theorem}[Bezout's Theorem]
    Let \(a, b \in \mathbb{Z}\) such that one of them is not zero. If the \(\gcd(a, b) = 1\), then there are integers \(x, y \in \mathbb{Z}\) such that \(ax + by = 1\).
\end{theorem}

\begin{proof}
    Let \(S = \left\{ax + by \in \mathbb{Z}^+ : x, y \in \mathbb{Z}\right\}\). Without loss of generality, we assume that \(a \neq 0\). Choose \(x = a\) and \(y = 0\). Thus, we have that \(ax + by = a^2 > 0\). Consequently, we have that \(S \neq \emptyset\) and that \(S \subseteq \mathbb{Z}^+\).

    By the WOA, there is the least element \(s \in S\). We (exercise) will show that \(s = \gcd(a, b) = 1\).
\end{proof}

\begin{theorem}[General Bezout]
    For \(a, b \in \mathbb{Z}\), there are \(x, y \in \mathbb{Z}\) such that \(ax + by = \gcd(a, b)\).
\end{theorem}

\begin{proof}
    The arguments are pretty similar to the previous case but a bit harder.
\end{proof}

\begin{claim}
    \((U(n), {}\cdot{})\) is a group.
\end{claim}

\begin{proof}
    \phantom{wow}

    \textit{\underline{Closure.}} We want to show that \(U(n)\) is closed under multiplication. Let \([a], [b] \in U(n)\). By construction of \(U(n)\), there are \([a'], [b'] \in \mathbb{Z}_n\) such that \([a] \cdot [a'] = [1]\) and \([b] \cdot [b'] = [1]\). We now want to show that \([a] \cdot [b] = [ab] \in U(n)\). Consider
    \[
    \begin{aligned}
        [ab] \cdot [a'b'] &= [(ab) \cdot (a'b')] \\
                          &= [(aa') \cdot (bb')] \\
                          &= [1 \cdot 1] = [1]
    \end{aligned}
    \]
    Thus, \([a] \cdot [b] \in U(n)\). Note that the commutativity on the second line derives from the commutativity of integer multiplication.

    \textit{\underline{Associativity.}} Consider
    \[
    \begin{aligned}
        ([a] \cdot [b]) \cdot [c] &= ([ab]) \cdot [c] \\
                                  &= [abc] \\
                                  &= [a] \cdot [bc] \\
                                  &= [a] \cdot ([b] \cdot [c])
    \end{aligned}
    \]

    \textit{\underline{Identity.}} Notice that \([1] \in U(n)\) is the identity.

    \textit{\underline{Inverse.}} Let \([a] \in U(n)\). We want to show that there is an inverse \([a]^{-1}\) for \([a]\). Since we construct \(U(n)\) by only picking elements of \(\mathbb{Z}_n\) that has an inverse also in \(\mathbb{Z}_n\), we can pick \([b] \in \mathbb{Z}\) such that \([a] \cdot [b] = [1]\). Since both \([a]\) and \([b]\) have an inverse in \(\mathbb{Z}_n\) and inverses are unique, then they must both be included in \(U(n)\). Thus, the inverse for \([a] \in U(n)\) is also in \(U(n)\) as well.
\end{proof}

\begin{theorem}[Uniqueness of Inverse]
    Let \(G\) be a group and \(a \in G\). There is a unique inverse of \(a\)
\end{theorem}

\begin{proof}
    Let \(a \in G\). Since \(G\) is a group, \(a\) must have an inverse, say \(b \in G\). Suppose that there is another inverse \(b' \in G\) of \(a\). We want to show that \(b' = b\). Since \(b\) and \(b'\) are inverses of \(a\), we have
    \[
        ab = e = ab'
    \]
    where \(e\) is the identity element. Consider
    \[
    \begin{aligned}
        ab &= ab' \\
        bab &= bab' \\
        eb &= eb' \\
        b &= b'
    \end{aligned}
    \]
\end{proof}

\begin{theorem}[Cancellation]
    Let \(a, b, c \in G\) such that \(ab = ac\). Then, \(b = c\).
\end{theorem}

\begin{proof}
    Assume that \(ab = ac\). Let \(a^{-1}\) denote the inverse of \(a\) and \(e\) denotes the identity element. Consider
    \[
    \begin{aligned}
        ab &= ac \\
        a^{-1}(ab) &= a^{-1}(ac) \\
        (a^{-1}a)b &= (a^{-1}a)c \\
        eb &= ec \\
        b &= c
    \end{aligned}
    \]
\end{proof}

\begin{theorem}[Socks-Shoes Property]
    For \(a, b \in G\) where \(G\) is a group. We have that
    \[
        (ab)^{-1} = b^{-1}a^{-1}
    \]
\end{theorem}

\begin{proof}
    Observe that
    \[
        (ab)(ab)^{-1} = e \\
    \]
    Consider
    \[
    \begin{aligned}
        (ab)(b^{-1}a^{-1}) &= a(bb^{-1})a^{-1} \\
                           &= aea^{-1} = aa^{-1} \\
                           &= e
    \end{aligned}
    \]
    We now have
    \[
        \cancel{(ab)}(ab)^{-1} = \cancel{(ab)}(b^{-1}a^{-1})
    \]

    Therefore, we have that
    \[
        (ab)^{-1} = b^{-1}a^{-1}
    \]
\end{proof}

\begin{nexample}
    We want to show that for any \(a, b \in G\) where \((ab)^2 = a^2b^2\) implies that \(ab = ba\). That is, to show that \(G\) is Abelian.

    \begin{proof}
        Let \(a, b \in G\) such that \(ab)^2 = a^2 b^2\). This is the same as
        \[
        \begin{aligned}
            (ab)(ab) &= a \cdot a \cdot b \cdot b \\
            abab &= aabb \\
            a^{-1}(abab)b^{-1} &= a^{-1}(aabb)b^{-1} \\
            (a^{-1}a)(ba)(bb^{-1}) &= (a^{-1}a)(ab)(bb^{-1}) \\
            ba &= ab
        \end{aligned}
        \]
    \end{proof}
\end{nexample}
