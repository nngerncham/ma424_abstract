\chapter{External Direct Product}

\section{Definitions and Examples}

\begin{definition}[External direct product]
    Let \(G_1, G_2, \ldots, G_n\) be groups. The \textit{external direct product} of \(G_1, G_2, \ldots, G_n\) is
    \[
        G_1 \circledplus G_2 \circledplus \cdots \circledplus G_n = 
        \left\{(g_1, g_2, \ldots, g_n) : g_i \in G_i\right\}
    \]
    with respect to each group operation.
\end{definition}

\begin{nexample}
    Consider \(\mathbb{R}^2\). In this context, we can define the following with ``\(+\)'':
    \[
        \mathbb{R}^2 = \mathbb{R} \circledplus \mathbb{R}
    \]

    Similarly,
    \[
    \begin{aligned}
        \mathbb{C} &= \mathbb{R} \circledplus \mathbb{R} \\
                   &= \left\{x + iy : x, y \in \mathbb{R}\right\} \\
                   &\cong \left\{(x, y) : x, y \in \mathbb{R}\right\}
    \end{aligned}
    \]
\end{nexample}

\begin{nexample}
    Consider \(\mathbb{Z}_3 \circledplus \mathbb{Z}_2\) under \(+\). We would have that
    \[
    \begin{aligned}
        \mathbb{Z}_3 \circledplus \mathbb{Z}_2 
            &= \left\{(0, 0), (0, 1), (1, 0), (1, 1), (2, 0), (2, 1)\right\}
    \end{aligned}
    \]
    In computation, we have
    \[
    \begin{aligned}
        (0, 1) \circledplus (2, 1) &= (0+2, 1+1) \\
                                   &= (2, 2) \\
                                   &= (2, 0)
    \end{aligned}
    \]
\end{nexample}

\begin{nexample}
    Consider \(U(3) \circledplus U(4)\). Recall that
    \[
    \begin{aligned}
        U(3) &= \left\{1, 2\right\} \\
        U(4) &= \left\{1, 3\right\}
    \end{aligned}
    \]
    Thus, 
    \[
        U(3) \circledplus U(4) = \left\{(1, 1), (1, 3), (2, 1), (2, 3)\right\}
    \]
    In computation, we have
    \[
    \begin{aligned}
        (1, 3) \circledplus (2, 1) &= (1 \cdot 2, 3 \cdot 1) \\
                                   &= (2, 3)
    \end{aligned}
    \]
\end{nexample}

\begin{nexample}
    Let \(G\) be a group of order 4. Observe that \(G \cong \mathbb{Z}_4\) or \(G \cong \mathbb{Z}_2 \circledplus \mathbb{Z}_2\). The second clause comes from the fact that
    \[
        \mathbb{Z} \circledplus \mathbb{Z} = \left\{(0, 0), (0, 1), (1, 0), (1, 1)\right\}
    \]
\end{nexample}

\section{Applications of EDP}

\begin{theorem}[Order of EDP]
    Let \(G_1, G_2, ..., G_n\) be groups and \((g_1, g_2, ..., g_n) \in G_1 \circledplus G_2 \circledplus \cdots \circledplus G_n\). We have that
    \[
        |(g_1, g_2, ..., g_n)| = \lcm(|g_1|, |g_2|, ..., |g_n|)
    \]
\end{theorem}
