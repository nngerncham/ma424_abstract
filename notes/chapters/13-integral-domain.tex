\chapter{Integral Domain}

\begin{definition}[Zero Divisor]
    Let \(R\) be a ring. An element \(a \in R\) is called a \textit{zero divisor} if \(a \neq 0\) and there is \(b \neq 0 \in R\) such that \(ab = 0\) or \(ba = 0\).
\end{definition}

\begin{definition}[Integral Domain]
    An \textit{integral domain} is a commutative ring with unity and no zero divisors.
\end{definition}

\begin{theorem}
    Let \(a, b, c \in R\) where \(R\) is an integral domain. Then, \(ab = ac\) implies \(b = c\).
\end{theorem}

\begin{definition}[Field]
    A field \(F\) is a ring such that
    \begin{enumerate}
        \item \(F\) has a unity
        \item \(F\) is commutative
        \item For all \(x \in F \setminus \left\{0\right\}\), \(x\) has a multiplicative inverse
    \end{enumerate}
\end{definition}

\begin{remark}
    A field \((F, +, \cdot)\) is a commutative ring such that
    \begin{enumerate}
        \item \((F, {}+{})\) is an Abelian group
        \item \((F \setminus \left\{0\right\}, {}\cdot{})\) is an Abelian group
    \end{enumerate}
\end{remark}

\begin{theorem}
    A finite integral domain is a field.
\end{theorem}
