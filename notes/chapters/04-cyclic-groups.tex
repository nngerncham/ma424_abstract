\chapter{Cyclic Groups}

\section{Definitions}

\begin{definition}[Cyclic Group]
    \(G\) is a \textit{cyclic group} if there is \(a \in G\) such that \(G = \left\{a^n : n \in \mathbb{Z}\right\} = \langle a \rangle\). Additionally, \(a \in G\) is called a \textit{generator} of \(G\).

    Notation-wise, we have the following:
    \[
    \begin{aligned}
        n \geq 1 &: a^n = \underbrace{a a \cdots a}_\text{\(n\) times} \\
        n = 0 &: a^0 = e \\
        n = -k, k \in \mathbb{N} &: a^{n} = (a^{-1})^k = \underbrace{a^{-1} a^{-1} \cdots a^{-1}}_\text{\(n\) times}
    \end{aligned}
    \]
\end{definition}

\begin{nexample}
    Consider the group \((\mathbb{Z}, +)\). Let \(n \in \mathbb{Z}\) and consider \(1 \in \mathbb{Z}\).
    \[
    \begin{aligned}
        n \geq 1 &: n = \underbrace{1+1+\cdots+1}_\text{\(n\) times} \\
        n = 0 &: 1^0 = e = 0 \\
        n \leq -1, n=-k, k \in \mathbb{N} &: 1^{-k} = (-1)^k = -k
    \end{aligned}
    \]
    Thus, \((\mathbb{Z}, +) = \langle 1 \rangle\). Similarly, we can show that \((\mathbb{Z}, +) = \langle -1 \rangle\) as well.
\end{nexample}

\begin{remark}
    If \(G = \langle a \rangle\), then \(G = \langle a^{-1} \rangle\) as well.
\end{remark}

\begin{remark}
    If \(G\) is Abelian, we can use the \textit{additive} version of \(x^n = x+x+\cdots+x = nx\).
\end{remark}

\begin{nexample}
    Consider \((\mathbb{Z}_6, +)\). Notice that
    \[
    \begin{aligned}
        \mathbb{Z}_6 &= \left\{[0], [1], [2], [3], [4], [5]\right\} \\
                     &= \langle [1] \rangle \\
                     &= \langle [-1] \rangle \\
                     &= \langle [5] \rangle 
    \end{aligned}
    \]

    Are there any other generators? (Hint: no, but you can try them out first)
    \[
    \begin{aligned}
        \langle [0] \rangle &= [0] \\
        \langle [2] \rangle &= \left\{[0], [2], [4]\right\} \\
        \langle [3] \rangle &= \left\{[0], [3]\right\} \\
        \langle [4] \rangle &= \left\{[0], [2], [4]\right\} \\
        \langle [5] \rangle &= \left\{[0], [1], [2], [3], [4], [5]\right\}
    \end{aligned}
    \]

    Conventionally, we can say that
    \[
        \mathbb{Z}_6 = \langle [1] \rangle = \langle [5] \rangle
    \]
\end{nexample}

Notice that, in \(\mathbb{Z}_n\), \(\langle [a] \rangle = \left\{k[a] : k \in \mathbb{Z}\right\}\). We claim that \(\langle a \rangle = \mathbb{Z}_n\) if for any \([x] \in \mathbb{Z}_n\), there is \(k \in \mathbb{Z}\) such that \([x] = k[a] = [ka]\). In other words, \(x - ka\) is divisible by \(n\).

\subsection{Bezout's Theorem on Additive Integers with Modulo \(n\)}

\begin{nexample}
    Consider \(\mathbb{Z}_{14} = \left\{[0], [1], \ldots, [13]\right\}\). Notice that
    \[
    \begin{aligned}
        \langle [0] \rangle &= \left\{k[0] : k \in \mathbb{Z}\right\} \\
                            &= \left\{[0]\right\} \\
        \langle [1] \rangle &= \mathbb{Z}_{14} \\
        \langle [2] \rangle &= \left\{k \cdot [2] : k \in \mathbb{Z}\right\} \\
                            &= \left\{[2], [4], [6], [8], [10], [12]\right\} \\
        \langle [3] \rangle &= \left\{k[3] : k \in \mathbb{Z}\right\} \\
                            &= \left\{[0], [3], [6], [9], [12], [15] = [1], [4], [7], [10], [13], [16]=[2], [5], [8], [11], [14]=[0]\right\} \\
                            &= \mathbb{Z}_{14} \\
                            &\phantom{|.}\vdots
    \end{aligned}
    \]
    Notice that we are unable to generate \([14]\) using \([2]\) but we can do so with \([1]\) and \([3]\).
\end{nexample}

The previous example might now make us wonder: \textit{What are the \([s]\)'s such that \(\langle [s] \rangle\) generates \(\mathbb{Z}_{14}\)}. That is, we want to find \([s]\) such that
\[
\begin{aligned}
    \langle [s] \rangle = \left\{k[s] : k \in \mathbb{Z}\right\} = \mathbb{Z}_{14}
\end{aligned}
\]

Since we already know that \([1]\) can always generate \(\mathbb{Z}_{14}\), we can narrow down our search. Namely, we want to find \(k \in \mathbb{Z}\) such that
\[
    k[s] = [1] \iff 14 \mid ks-1
\]
In other words, there is \(l \in \mathbb{Z}\) such that \(ks-1 = 14l\). Consider
\[
\begin{aligned}
    1 &= 14(-l) + s(-k) \\
    1 &= 14l_1 + sk_1 &\quad \text{where } l_1 = -l, k_1 = -k
\end{aligned}
\]

Recall Theorem \ref{thm:bezout}: If \(a, b \in \mathbb{Z} \setminus \left\{0\right\}\), then there are \(x, y \in \mathbb{Z}\) such that \(ax + by = \gcd(a, b)\). Therefore, we have that we can find \(l_1, k_1 \in \mathbb{Z}\) that satisfies the above equation if and only if \(\gcd(14, s) = 1\). In conclusion, we have that
\begin{enumerate}
    \item \(\mathbb{Z}_{14} = \langle [s] \rangle\) if and only if \(\gcd(14, s) = 1\)
    \item In general, \(\mathbb{Z}_n = \langle [s] \rangle\) if and only if \(\gcd(n, s) = 1\)
\end{enumerate}

\subsection{Bezout's Theorem on Multiplicative Integers with Modulo \(n\)}

Recall the definition of the multiplicative integer with modulo \(n\):
\[
    U(n) = \left\{[k] \in \mathbb{Z}_n : \ \exists \ [s] \in \mathbb{Z}_n, [k][s] = [1]\right\}
\]
The last part simply says that \([k]\) has an inverse.

\begin{nexample}
    Does \([3]\) have an inverse in \(U(n)\)? Consider
    \[
    \begin{aligned}
        [3][7] &= [3 \cdot 7] \\
               &= [21] \\
               &= 2[10] + [1] \\
               &= [1]
    \end{aligned}
    \]
    Thus, \([7]\) is the inverse of \([3]\) in \(U(10)\).

    What about \([2]\)? First, we can see that \(\gcd(2, 10) = 2\) and that we can never achieve \(2x + 10y = 1\) for any \(x, y \in \mathbb{Z}\) by Bezout's Theorem. Thus, \([2]\) does not have an inverse.
\end{nexample}

From the previous example, we can observe the following:
\begin{enumerate}
    \item \([s] \in U(10)\) has a multiplicative inverse if and only if \(\gcd(s, 10) = 1\)
    \item In general, \([s] \in U(n)\) has an inverse if and only if \(\gcd(s, n) = 1\)
\end{enumerate}

Additionally, we can simplify the computation of the elements of the multiplicative integer with modulo \(n\) group to be as follows:
\[
    U(n) = \left\{[k] \in \mathbb{Z}_n : \gcd(k, n) = 1\right\}
\]

\begin{nexample}
    From our above results, we have that
    \[
        U(10) = \left\{[1], [3], [7], [9]\right\}
    \]
\end{nexample}

\begin{nexample}
    What about the generators for \(U(10)\)? Consider the following:
    \[
    \begin{aligned}
        \langle [1] \rangle &= \left\{[1]\right\} \\
        \langle [3] \rangle &= \left\{[3], [3]^2 = [9], [3]^3 = [27] = 7, [3]^4 = [81] = [1]\right\} \\
                            &= U(10) \\
        \langle [7] \rangle &= \left\{[7], [7]^2 = [49] = 9, [7]^3 = [63] = [3], [7]^4 = [1]\right\} \\
                            &= U(10) \\
        \langle [9] \rangle &= \left\{[9], [9]^2 = [81] = [1], [9]^3 = [729] = [9]\right\}
    \end{aligned}
    \]
    Since the generators eventually come back to \([1]\), we can say that \(U(10)\) is cyclic and can be generated by \(\langle [3] \rangle\) and \(\langle [7] \rangle\).
\end{nexample}

\begin{distraction}
    Another notation of the multiplicative integer with modulo \(n\) is
    \[
        U(n) = \left(\mathbb{Z}_n\right)^\times
    \]
\end{distraction}

\begin{nexample}
    Is \(U(8)\) cyclic? Notice that
    \[
        U(8) = \left\{[1], [3], [5], [7]\right\}
    \]

    We can try to compute the generators as follows:
    \[
    \begin{aligned}
        \langle [1] \rangle &= \left\{[1]\right\} \\
        \langle [3] \rangle &= \left\{[3], [3]^2 = [9] = 1\right\} \\
        \langle [5] \rangle &= \left\{[5], [5]^2 = [25] = [1]\right\} \\
        \langle [7] \rangle &= \left\{[7], [7]^2 = [49] = [1]\right\}
    \end{aligned}
    \]

    Thus, we can conclude that \(U(8)\) is not cyclic.
\end{nexample}

\begin{remark}
    \(U(n)\) is cyclic if
    \[
        n = 1, 2, 4, p^k \text{ or } 2p^k
    \]
    where \(p\) is an odd prime and \(k \in \mathbb{N}\).
\end{remark}

\section{Criterion for \(a^i = a^j\)}

\begin{theorem}[Criterion for \(a^i = a^j\)]
    Let \(G\) be a group and \(a \in G\).
    \begin{enumerate}
        \item If \(|a| = \infty\), then 
            \[
                a^i = a^j \iff i = j
            \]
        \item If \(|a| = n\), then
            \[
                a^i = a^j \iff n \mid i-j
            \]
    \end{enumerate}
\end{theorem}

\begin{proof}
    \phantom{wow}

    \underline{\textit{Part 1.}} Assume that \(|a| = \infty\). Then, there is no non-zero \(n \in \mathbb{Z}^+\) such that \(a^n = e\). That is, the two-way sequence \(\ldots, a^{-1}, e, a^1, \ldots\) never repeats. Thus, for any \(i, j \in \mathbb{Z}\),
    \[
    \begin{aligned}
        a^i = a^j &\iff a^{i-j} = a^{j-j} = a^{0} = e\\
                  &\iff i-j = 0 \\
                  &\iff i=j
    \end{aligned}
    \]

    \underline{\textit{Part 2.}} Assume that \(|a| = n \in \mathbb{N}\). Thus, \(n\) is the smallest positive integer such that \(a^n = e\). We will show that \(\langle a \rangle = \left\{e, a, \ldots, a^{n-1}\right\}\). Note that \(\langle a \rangle = \left\{a^k : k \in \mathbb{Z}\right\}\). We will do this by showing that at some point, \(a^k\) will start to have the same values.

    First, we will show that \(\langle a \rangle\) is finite. Let \(k \in \mathbb{Z}\). Since \(n \in \mathbb{N}\), by the division algorithm, there are \(q \in \mathbb{Z}\) and \(r \in \mathbb{Z}\) such that \(k = qn+r\) with \(0 \leq r \leq n-1\). Consider
    \[
    \begin{aligned}
        a^k &= a^{qn + r} \\
            &= a^{qn} a^r \\
            &= (a^n)^qa^r \\
            &= e^q a^r \\
            &= ea^r \\
        a^k &= a^r
    \end{aligned}
    \]
    Thus, we have that \(a^k \in \left\{e, a, \ldots, a^{n-1}\right\}\) for any \(k \in \mathbb{Z}\). Therefore, \(\langle a \rangle = \left\{e, a, a^2, \ldots, a^{n-1}\right\}\).

    Now, we will prove the forward case of the claim. Assume that \(a^i = a^j\). Then, we will have that
    \[
    \begin{aligned}
        a^{i} &= a^{j} \\
        a^i a^{-j} &= e \\
        a^{i-j} &= e
    \end{aligned}
    \]
    Since \(i, j \in \mathbb{Z}\), then so is \(i-j\). By the division algorithm, there are integers \(q, r \in \mathbb{Z}\) such that
    \[
    \begin{aligned}
        i-j = qn + r : 0 \leq r \leq n-1
    \end{aligned}
    \]
    Then, \(a^{i-j} = a^{qn + r}\). This implies that
    \[
    \begin{aligned}
        e &= a^{i-j} \\
          &= a^{qn+r} \\
          &= a^{qn} a^r \\
          &= ea^r = a^r
    \end{aligned}
    \]
    So, we have that \(a^r = e\) and \(0 \leq r \leq n-1\). Since \(n\) is the smallest natural (positive) integer such that \(a^n = e\), we have that \(r = 0\). Thus, \(i-j = qn\) and therefore, \(n \mid i-j\).

    Finally, we will prove the converse case of the claim. Assume that \(n \mid i-j\). Then, there are \(q \in \mathbb{Z}\) such that \(i-j = qn\). Thus, \(a^{i-j} = a^{qn} = e\).
\end{proof}

\begin{corollary}
    The order of an element is the cardinality of the the set of its generator. Symbolically,
    \[
        |a| = |\langle a \rangle|
    \]
\end{corollary}

\begin{corollary}
    If \(k\) is an integer such that \(a^k = e\), then \(k\) is a multiple of the order of \(a\). Symbolically,
    \[
        a^k = e \iff |a| \mid k
    \]
\end{corollary}

\begin{corollary}
    If \(a, b \in G\) and \(ab = ba\), then \(|ab| \mid |a||b|\).
\end{corollary}

\begin{proof}
    Exercise!
\end{proof}

\begin{remark}
    By the division algorithm and the Abelian structure of cyclic groups \((G = \langle a \rangle)\), they have the same structure as \(\mathbb{Z}_n\) for some \(n \in \mathbb{N}\) (including \(n = \infty\)). A fancy way to say this is that \(G = \langle a \rangle\) is \textit{isomorphic} to \(\mathbb{Z}\) or \(\mathbb{Z}_n\) for some \(n \in \mathbb{N}\).
\end{remark}

\begin{theorem}\label{thm:four-two}
    Let \(a \in G\) be such that \(|a| = n\). We have that
    \[
        \langle a^k \rangle = \left\langle a^{\gcd(n, k)} \right\rangle
    \]
    and
    \[
        \left|a^k\right| = \frac{n}{\gcd(n, k)}
    \]
\end{theorem}

\begin{proof}
    Suppose that \(|a| = n\). Let \(d = \gcd(n, k)\). This also means that \(k\) is a multiple of \(d\). We will now prove the first part of the theorem.

    Since \(a^k = a^{dr}\), we already have that \(\langle a^k \rangle \subseteq \langle a^d \rangle\). It now suffices to show that \(\langle a^d \rangle \subseteq \langle a^k \rangle\) as well. By Bezout's theorem, there are \(s, t \in \mathbb{Z}\) such that \(d = ns + kt\). Thus, we have that 
    \[
    \begin{aligned}
        a^d &= a^{ns + kt} \\
            &= a^{ns} a^{kt} \\
            &= (a^n)^s (a^k)^t \\
            &= e(a^k)^t \\
            &= (a^k)^t \in \langle a^k \rangle
    \end{aligned}
    \]
    Thus, we have that \(\langle a^d \rangle \subseteq \langle a^k \rangle\). Therefore, we can conclude that \(\langle a^k \rangle = \langle a^d \rangle\).

    We will now prove the second part of the theorem. First, claim that \(\left|a^c\right| = n/c\) for any divisor \(c\) of \(n\). Observe that
    \[
        (a^c)^{\frac{n}{c}} = a^n = e
    \]
    Thus, \(\left|a^c\right| \leq \frac{n}{c}\). Since \(|a| = n\), we have that \((a^c)^i \neq e\) for any \(i \in \mathbb{N}\) such that \(i < \frac{n}{c}\). Hence, it follows that \(\frac{n}{c} \in \mathbb{N}\) is the smallest positive integer such that \((a^c)^\frac{n}{c} = e\). Recall from the previous part that \(\langle a^k \rangle = \langle a^{\gcd(n, k)} \rangle\). Then, it follows that
    \[
        \left|\langle a^k \rangle\right| = \left|\langle a^{\gcd(n, k)} \rangle\right|
    \]

    Therefore, we can conclude that
    \[
    \begin{aligned}
        \left|a^k\right| &= \left|\langle a^k \rangle\right| \\
                         &= \left|\langle a^{\gcd(n, k)} \rangle\right| \\
                         &= \frac{n}{\gcd(n, k)}
    \end{aligned}
    \]
\end{proof}

\begin{nexample}
    Suppose that \(|a| = 36\) for some \(a \in G\). Let us try to find \(\left|\langle a^8 \rangle\right|\). Since \(n = 36\) and \(k = 8\), we will have that
    \[
    \begin{aligned}
        \left|\langle a^8 \rangle\right|
            &= \frac{n}{\gcd(n, k)} \\
            &= \frac{36}{\gcd(36, 8)} \\
            &= \frac{36}{4} = 9
    \end{aligned}
    \]
\end{nexample}

\begin{corollary}
    Let \(G\) be a finite cyclic group. For \(g \in G\), \(|g|\) divides \(|G|\).
\end{corollary}

\begin{corollary}
    Let \(|a| = n\). Then, we have that
    \begin{enumerate}
        \item \(\langle a^i \rangle = \langle a^j \rangle\) if and only if \(\gcd(n, i) = \gcd(n, j)\); and
        \item \(\left|a^i\right| = \left|a^j\right|\) if and only if \(\gcd(n, i) = \gcd(n, j)\).
    \end{enumerate}
\end{corollary}

\begin{corollary}
    Let \(|a| = n\). We have that
    \begin{enumerate}
        \item \(\langle a \rangle = \langle a^j \rangle\) if and only if \(\gcd(n, j) = 1\); and
        \item \(|a| = |a^j|\) if and only if \(\gcd(n, j) = 1\).
    \end{enumerate}
\end{corollary}

\begin{corollary}
    \[
        \mathbb{Z}_n = \langle [k] \rangle \iff \gcd(n, k) = 1
    \]
\end{corollary}

\section{Classification of Subgroups of Cyclic Groups}

\begin{theorem}[Fundamental Theorem of Cyclic Groups]
    Every subgroup of a cyclic group is cyclic. If \(\left|\langle a \rangle\right| = n\) and \(k \mid n\), then there is a unique subgroup of order \(k\)---namely, \(\langle a^{\frac{n}{k}} \rangle\).
\end{theorem}

\begin{proof}
    Let \(G = \langle a \rangle\) and \(H \leq G\). If \(H = \left\{e\right\}\), then \(H = \langle e \rangle\).

    Now, suppose that \(H \neq \left\{e\right\}\). Since \(H \leq G = \langle a \rangle\), every element of \(H\) is of the form \(a^t\) for some \(t \in \mathbb{Z}\). Let \(S = \left\{k \in \mathbb{N} : a^k \in H\right\}\). Note that \(a^{-t} \in H\) because \(H\) is a group. Since \(H \neq \left\{e\right\}\), we also have that \(S \neq \emptyset\). By the WOA, \(S\) has the smallest element, say \(m \in S\).

    Since \(a^m \in H\), we have that \(\langle a^m \rangle \subseteq H\). We will now show that \(H = \langle a^m \rangle\). For this, it suffices to show that \(H \subseteq \langle a^m \rangle\). Let \(b \in H\). Thus, \(b \in G\) as well. Since \(G\) is cyclic, \(b = a^k\) for some \(k \in \mathbb{Z}\). By the division algorithm, there are \(q, r \in \mathbb{Z}\) such that \(k = qm + r\) with \(0 \leq r \leq m-1\). Then, \(a^k = a^{qm+r}\). Consider
    \[
    \begin{aligned}
        a^k &= a^{qm}a^r \\
        a^{-qm}a^k &= a^{-qm}a^{qm}a^r \\
        a^{k - qm} &= a^r \in H
    \end{aligned}
    \]
    Since \(m\) is the smallest positive integer such that \(a^m \in H\) and \(0 \leq r < m\), we have that \(r = 0\). Thus, we have that \(b = a^k = a^{mq+r} = a^{mq} = (a^m)^q\). That is, an arbitrary \(b \in H\) is also a member of \(\langle a^m \rangle\). Hence, we have that \(H \subseteq \langle a^m \rangle\). Therefore, we can conclude that \(H = \langle a^m \rangle\).

    Now, let \(|G| = |\langle a \rangle| = n\) and \(k \in \mathbb{N}\) such that \(k \mid n\). We will show that there is a unique subgroup of \(G\) with order \(k\). Let us first show the existence of this subgroup. Choose \(H = \langle a^m \rangle\) where \(m = \frac{n}{k} \in \mathbb{N}\). Thus, we will have that
    \[
    \begin{aligned}
        |H| &= \left|\langle a^m \rangle\right| \\
            &= \frac{n}{\gcd(m, n)} \\
            &= \frac{n}{m} \\
            &= \frac{n}{\frac{n}{k}} \\
            &= k
    \end{aligned}
    \]
    Thus, we can conclude that we can always find a subgroup with order \(k\).

    Now, we will show that this subgroup is unique. Assume that there are subgroups \(H_1 = \langle a^{m_1} \rangle\) and \(H_2 = \langle a^{m_2} \rangle\) such that \(|H_1| = |H_2|\) where \(m_1, m_2 \in \mathbb{N}\). Consider
    \[
    \begin{aligned}
        |H_1| &= \left|\langle a^{m_1} \rangle\right| \\
              &= \frac{n}{\gcd(m_1, n)}
    \end{aligned}
    \]
    and
    \[
    \begin{aligned}
        |H_2| &= \frac{n}{\gcd(m_2, n)}
    \end{aligned}
    \]
    Then, we have that
    \[
    \begin{aligned}
        \frac{n}{\gcd(m_1, n)} &= \frac{n}{\gcd(m_2, n)} \\
        \gcd(m_1, n) &= \gcd(m_2, n)
    \end{aligned}
    \]
    Let \(g = \gcd(m_1, n) = \gcd(m_2, n)\). Then, by Theorem \ref{thm:four-two}, we have that \(\langle a^{m_1} \rangle = \langle a^{g} \rangle = \langle a^{m_2} \rangle\).

    Therefore, we can conclude that the subgroup \(H \leq G\) with order \(k\) such that \(k \mid n = |G|\) is unique.
\end{proof}

\begin{nexample}
    Consider \(G = \mathbb{Z}_{30} = \langle 1 \rangle = \langle 7 \rangle = \langle 11 \rangle = \cdots\). Let us narrow down slightly and let \(G = \langle a \rangle\) for some \(a \in \mathbb{Z}_{30}\). Notice that
    \[
        \langle a \rangle = \left\{[0], [1], \ldots, [29]\right\}
    \]
\end{nexample}

\begin{corollary}[Subgroups of \(\mathbb{Z}_n\)]
    For any positive integer \(k \mid n\), we will have that \(\left\langle \left[\frac{n}{k}\right] \right\rangle\) is the unique subgroup of \(\mathbb{Z}_n\) of order \(k\).
\end{corollary}

\begin{definition}[Euler \(\phi\) Function]
    Let \(n \in \mathbb{N}\). The \textit{Euler Phi Function} \(\phi(n)\) is the number of coprimes \(i\) of \(n\) such that \(1 \leq i \leq n\).
\end{definition}

\begin{nexample}
    \[
    \begin{aligned}
        \phi(1) &= 1 \\
        \phi(n) &= \sum_{i=1}^n \mathds{1}\left[\gcd(i, n) = 1\right]
        \phi(p); \text{\(p\) prime} &= p-1
    \end{aligned}
    \]
\end{nexample}

\begin{theorem}
    If \(d\) is a positive divisor of \(n\), the number of elements of order \(d\) in the cyclic group with order \(n\) is \(\phi(d)\).
\end{theorem}

\begin{proof}
    By Theorem \ref{thm:four-two} and the FToCG.
\end{proof}
