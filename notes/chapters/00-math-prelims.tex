\chapter{Mathematical Preliminaries}

\section{Introduction}

\subsection{What is Algebra}

What's the difference between algebra and arithmetic? An easy way to differentiate them is this: arithmetics deal with numbers, algebra deals with the \textit{structure} of operations or numbers.

Now, onto the subject name---abstract algebra. What exactly does \textit{abstract} mean? Abstract, in this case, refers to structures. Abstract Algebra, then, is the study of said structure in hopes of simplifying it for further computations. Abstract Algebra is also known as Modern Algebra as it founded quite recently in the 1800s.

\subsection{Sets}

Using set theory, we are able to recursively define the integers as follows:
\[
\begin{aligned}
    0 &= \emptyset = \text{ empty set} \\
    1 &= \left\{\emptyset\right\} = \left\{0\right\} \\
    2 &= \left\{0, 1\right\} \\
    3 &= \left\{0, 1, 2\right\} \\
      &\phantom{0}\vdots \\
    S^+ &= S \cup \left\{S\right\} = S+1 &\text{(\(S^+\) is the \textit{successor} of \(S\))}
\end{aligned}
\]
We also define the set of natural numbers \(\mathbb{N}\) to be \( \mathbb{N} = \mathbb{Z}^+ = \left\{1, 2, 3, \ldots\right\}\).

\section{Well-Ordering Axiom and Induction}

\begin{theorem}[Well-Ordering Axiom (WOA)]
    If \(S \subseteq \mathbb{Z}^+\) and \(S \neq \emptyset\), then \(S\) has the least, say \(a \in S\).
\end{theorem}

\begin{distraction}
    Note that we can use WOA for the the sets \(\mathbb{Z}^+ \cup \left\{0\right\}\) or \(\mathbb{Z}^+ \cup \left\{-1, 0\right\}\) as well.
\end{distraction}

\begin{theorem}[Principal of Mathematical Induction 1 or Induction]
    For every \(n \in \mathbb{Z}^+\), let \(T(n)\) be a statement. If the following conditions hold:
    \begin{enumerate}
        \item \(T(1)\) is true
        \item For all \(k \in \mathbb{Z}^+\), \(T(k)\) being true implies that \(T(k+1)\) is true
    \end{enumerate}
    Then, \(T(n)\) is true for all \(n \in \mathbb{Z}^+\).
\end{theorem}

\begin{proof}
    Assume that \(T(n)\) satisfies both conditions for each \(n \in \mathbb{Z}^+\). Let \(S = \left\{n \in \mathbb{Z}^+ : T(n) \text{ is false}\right\}\) which is a subset of \(\mathbb{Z}^+\). If \(S = \emptyset\), then we are done.

    Otherwise, assume to the contrary that \(S \neq \emptyset\). By the WOA, \(S\) has the least element \(a \in S\). By condition 1, we have that \(T(1)\) must be true. Thus, we have that \(1 \notin S\) and \(a > 1\). Let \(k = a - 1 \in \mathbb{Z}^+\). Since \(a\) is the smallest element of \(S\), we have that \(k \notin S\). That is, \(T(k)\) is true. Then, by condition 2, \(T(k+1) = T(a)\) must be true as well. This is a contradiction.

    Therefore, \(S\) must be empty.
\end{proof}

\begin{theorem}[Principal of Mathematical Induction 2 or Strong Induction]
    For each \(n \in \mathbb{Z}^+\), let \(T(n)\) be a statement. If the following conditions hold:
    \begin{enumerate}
        \item \(T(1)\) is true
        \item For every \(k \in \mathbb{Z}^+\), \(T(1), T(2), \ldots, T(k)\) being true implies that \(T(k+1)\) is true
    \end{enumerate}
    Then, \(T(n)\) is true for all \(n \in \mathbb{Z}^+\).
\end{theorem}

\begin{proof}
    Exercise!
\end{proof}

\subsection{Tower of Hanoi}

Recall the Tower of Hanoi game where there is a pattern to the number of steps needed to solve the game with \(n\) disks. We want to prove this pattern using PMI 1.

First, let us find the pattern.
\begin{itemize}
    \item with 1 disk, we take 1 step
    \item with 2 disks, we take 3 steps
    \item with 3 disks, we take 7 steps
    \item and so on...
\end{itemize}

Now, we can define the mathematical statement to be used with induction.

\begin{claim}[Tower of Hanoi]
    It takes \(2^n - 1\) steps to solve the Tower of Hanoi game with \(n\) disks.
\end{claim}

\begin{proof}
     Let \(T(n)\) be the statement ``It takes \(2^n - 1\) steps to solve the Tower of Hanoi game with \(n\) disks.''

    First, let us show that the base case holds. That is, let us show that it takes 1 step to solve the Tower of Hanoi with 1 disk. To solve this, we can simply move the disk from the first to the third pole---taking 1 step. Thus, the base case is proven.

    Now, let us show that the inductive case holds. Let us assume that \(T(k)\) holds and use it to show that \(T(k+1)\) holds as well. That is, it takes \(2^k-1\) steps to solve the Tower of Hanoi with \(k\) disks makes it so that it takes \(2^{k+1} - 1\) steps to solve the Tower of Hanoi with \(k+1\) disks.

    In order to solve the Tower of Hanoi with \(k+1\) disks, we need to do the following. First, we move the top \(k\) disks to the second pole. By the inductive assumption, this takes \(2^k - 1\) steps. Then, we move the bottom disk to the third pole which takes 1 step. Finally, we need to move the disks in the second pole to the third pole, finishing the game. By the inductive assumption, this also takes \(2^k - 1\) steps. In summary, this takes
    \[
    \begin{aligned}
        (2^k - 1) + 1 + (2^k - 1) &= 2 \cdot 2^k - 2 + 1 \\
                                  &= 2^{k+1} - 1 \text{ steps}
    \end{aligned}
    \]

    Therefore, by the Principal of Mathematical Induction I, we can conclude that it takes \(2^n - 1\) steps to solve the Tower of Hanoi with \(n\) disks.
\end{proof}

\section{Divisibility, GCD, Primes, and Unique Factorization}

\begin{definition}[Divisibility]
    Suppose that \(a, b \in \mathbb{Z}\). We say that \(a\) \textit{divides} \(b\), denoted as \(a \mid b\), if there exists \(c \in \mathbb{Z}\) such that \(b = ac\). Equivalently, we can say that
    \begin{itemize}
        \item \(a\) is a divisor of \(b\)
        \item \(b\) is a multiple of \(a\)
        \item \(b\) is divislble by \(a\)
    \end{itemize}
\end{definition}

\begin{definition}[Prime Numbers]
    An integer \(p > 1\) is \textit{prime} if for all \(a, b \in \mathbb{Z}\), then \(a = \pm 1\) or \(b = \pm 1\).
\end{definition}

\begin{distraction}
    There are 25 prime numbers that are less than or equals to 100.
\end{distraction}

\begin{theorem}[Unique Factorization or Division Algorithm]
    For any non-negative integers \(a\) and \(b\) with \(b > 0\), there exists a unique pair of integers \(q\) (quotient) and \(r\) (remainder) such that
    \[ a = bq + r \]
    where \(0 \leq r \leq b-1\).
\end{theorem}

\begin{proof}
    Let \(a, b \in \mathbb{Z}^+\left\{0\right\}\) be given. First, let us show the existence of their factorization. Let \(S = \left\{k \in \mathbb{Z}^+ \cup \left\{0\right\} : k = a - bn \ \exists \ n \in \mathbb{Z}\right\}\). If we let \(n = 0\), we will have that \(k = a - b(0) = a \in \mathbb{Z}^+ \cup \left\{0\right\}\) and that \(k \in S\). Thus, \(S \neq \emptyset\). By the Well-Ordering Axiom, \(S\) has the least element, say, \(r \in S\). Thus, \(r = a - bq \geq 0\) for some \(q \in \mathbb{Z}\). Claim that \(0 \leq r \leq b-1\).

    Assume to the contrary that \(r \geq b\). Consider the following:
    \[
    \begin{aligned}
        r' &= a - b(q + 1) \\
           &= \underbrace{a - bq}_{r} - b \\
           &= r - b \geq 0 \qquad \text{(by assumption that \(r \geq b\))}
    \end{aligned}
    \]
    Thus, \(r' \in S\). However, \(r' < r\) since \(b \geq 1\). This is a contradiction. Thus, the factorization of integers exists.

    Now, let us show the uniqueness of said factorization. Assume that there are \(q_1, q_2 \in \mathbb{Z}\) and \(r_1, r_2 \in \mathbb{Z}^+ \cup \left\{0\right\}\) such that \(a = bq_1 + r_1 = bq_2 + r_2\) where \(0 \leq r_1, r_2 \leq b-1\). Without loss of generality, assume that \(r_2 \geq r_1\). We now have the following:
    \[
    \begin{aligned}
        0 &= bq_2 + r_2 - (bq_1 + r_1) \\
          &= b(q_2 - q_1) + (r_2 - r_1) \\
        -(r_2 - r_1) &= b(q_2 - q_1) \\
    \end{aligned}
    \]

    Since \(0 \leq r_1, r_2 \leq b-1\), it follows that \(0 \leq r_2 - r_1 \leq b-1\). From our relation, we also have that \(b \mid (r_2 - r_1)\). Consequently, \(r_2 - r_1 = 0\) since \(b\) must divide it but their difference is at most \(b-1\). Since \(0 = b(q_2 - q_1)\), it must follows that either \(b\) or \(q_2 - q_1\) is 0. However, we already made the assumption that \(b > 0\), so it must be that \(q_2 - q_1 = 0\).

    Therefore, we can conclude that for \(q_1 = q_2\) and \(r_1 = r_2\) and that the factorization is unique.
\end{proof}

\begin{definition}[Greatest Common Divisor]
    The \textit{greatest common divisor (GCD)} is the number \(d \in \mathbb{Z}^+\) of two non-zero integers \(a, b\), written as \(d = \gcd(a, b)\) if \(d\) satisfies these two properties:
    \begin{enumerate}
        \item \(d \mid a\) and \(d \mid b\)
        \item For every \(c \in \mathbb{Z}\) such that \(c \mid a\) and \(c \mid b\), we have that \(d \geq c\)
    \end{enumerate}
\end{definition}

\begin{definition}[Relative Primes]
    If \(\gcd(a, b) = 1\), we say that \(a\) and \(b\) are relatively prime.
\end{definition}

\subsection{Euclidean Algorithm}

\begin{nexample}
    Let \(a = 101\) and \(b = 77\). We want to find \(\gcd(101, 77)\). Observe that 
    \[
    \begin{aligned}
        101 &= 77(1) + 24 &\quad 0 \leq 24 \leq 76 \\
    \end{aligned}
    \]
    Then, we set 77 to be the new \(a\) and 24 to be the new \(b\) and do the same thing over and over again. We will now have that
    \[
    \begin{aligned}
        77 &= 24(3) + 5 &\quad 0 \leq 5 \leq 23 \\
        24 &= 5(4) + 4 &\quad 0 \leq 4 \leq 4 \\
        5 &= 4(1) + 1 &\quad 0 \leq 1 \leq 3 \\
        4 &= 4(1) + 0 &\quad 0 \leq 0 \leq 1
    \end{aligned}
    \]

    Finally, we can conclude that \(\gcd(101, 77) = 1\).
\end{nexample}

\begin{theorem}[Euclidean Algorithm]
    Suppose that we are given \(a, b \in \mathbb{Z}\) that we want to find the greatest common denominator of. Without loss of generality, assume that \(0 < b < a\). Apply the Division Algorithm recursively as follows:
    \[
    \begin{aligned}
        r_0 &= a \\
        r_1 &= b \\
        r_{i} &= r_{i+1} \cdot q_{k+1} + r_{i+2}
    \end{aligned}
    \]
    with \(0 < r_{i+2} < r_{i+1}\) for \(0 \leq i < n - 1\) and \(r_{n+1} = 0\). Once terminated, we will have that \(r_n = \gcd(a, b)\).
\end{theorem}

\begin{proof}
    Can be done by using WOA.
\end{proof}

\section{Equivalence Relations}

\begin{definition}[Ordered Pair]
    Let \(a\) and \(b\) be sets. An ordered pair is
    \[
        (a, b) = \left\{\left\{a\right\}, \left\{a, b\right\}\right\}
    \]
\end{definition}

\begin{definition}[Cartesian Product]
    Let \(A\) and \(B\) be sets. The \textit{Cartesian Product} of \(A\) and \(B\) is
    \[
        A \times B = {(a, b) : a \in A, b \in B}
    \]
\end{definition}

\begin{nexample}
    Let \(A = \left\{0, 1, 2\right\}\) and \(B = \left\{a, b\right\}\). Then, we have
    \[
        A \times B = \left\{(0, a), (0, b), (1, a), (1, b), (2, a), (2, b)\right\}
    \]
    Intuitively, the cartesian product is the set of all possible combinations of elements of \(A\) and \(B\).
\end{nexample}

\begin{definition}[Relation]
    A subset of \(A \times B\) is called a \textit{relation} from \(A\) to \(B\). That is, \(R\) is a relation from \(A\) to \(B\) if and only if \(R \subseteq A \times B\). If \(R\) is a relation from \(A\) to \(A\), then \(R\) is called a relation on \(A\).
\end{definition}

\begin{definition}[Equivalence Relation]
    Let \(R\) be a relation on a set \(A\). \(R\) is called an \textit{equivalence relation} if the following properties hold
    \begin{enumerate}
        \item \textit{Reflexivity} --- For all \(a \in A\), \((a, a) \in R\)
        \item \textit{Symmetry} --- For all \(a, b \in A\), \((a, b) \in R\) implies that \((b, a) \in R\)
        \item \textit{Transitivity} For all \(a, b, c \in A\), \((a, b) \in R\) and \((b, c) \in R\) implies that \((a, c) \in R\)
    \end{enumerate}
\end{definition}

\begin{remark}
    If \(R\) is an equivalence relation, we write \(a \sim b\) to represent \((a, b) \in R\).
\end{remark}

\begin{nexample}
    Let \(A = \mathbb{Z}\). We define
    \[
        R_2 = \left\{(a, b) \in A \times A : 2 \mid a-b\right\}
    \]
    Let us show that \(R_2\) is a equivalence relation.
    \begin{enumerate}
        \item \textit{Reflexive} --- Let \(a \in \mathbb{Z}\). Then, \(a - a = 0\) and we have that \(2 \mid 0\). Thus, \((a, a) \in R_2\).
        \item \textit{Symmetry} --- Let \(a, b \in \mathbb{Z}\) and \((a, b) \in R_2\). We have that \(2 \mid a-b\). That means that there is \(k \in \mathbb{Z}\) such that \(a - b = 2k\). Therefore, \(b - a = 2(-k)\) and \(-k \in \mathbb{Z}\). We also have that \(2 \mid b-a\). Therefore, \((b, a) \in R_2\).
        \item \textit{Transitivity} --- Assume that \((a, b)\) and \((b, c) \in R_2\). Thus, \(2 \mid a-b\) and \(2 \mid b-c\). Therefore, there are \(k_2\) and \(k_3 \in \mathbb{Z}\) such that \(a-b = 2k_2\) and \(b-c = 2k_3\). Consider
            \[
            \begin{aligned}
                (a - b) + (b - c) &= 2k_2 + 2k_3 \\
                (a - c) &= 2(k_2 + k_3)
            \end{aligned}
            \]
            Note that \(k_2 + k_3 \in \mathbb{Z}\). Therefore, we have that \((a, c) \in R_2\).
    \end{enumerate}
\end{nexample}

\begin{definition}[Congruency]
    Let \(n \in \mathbb{Z}^+\) and \(a, b \in \mathbb{Z}\). If \(n \mid a-b\), then we write
    \[
        a \equiv b \pmod n
    \]
    That is, \(a\) is \textit{congruence} to \(b\).
\end{definition}

\begin{remark}
    Note that the \(\pmod n\) says that the modulo is applied on \textit{both} sides of the equation. Intuitively, \(a\) is congruent to \(b\) if \(a \mod n = b \mod n\).
\end{remark}

\begin{nexample}
    \phantom{wow}

    \begin{itemize}
        \item ``='' is an equivalence relation
        \item \((\mathbb{Z}, <)\) is not an equivalence relation since it fails reflexivity and symmetry (but is a type relation)
        \item \((\mathbb{Z}, \leq)\) is not an equivalence relation since it fails symmetry
    \end{itemize}
\end{nexample}

\begin{definition}[Partition]
    Define the set \(P\) as follows for some index set \(\Lambda\):
    \[
        P = \left\{A_i : A_i \subseteq S, i \in \Lambda\right\}
    \]
    We say that \(P\) is a partition of \(S\) if the following conditions satisfy:
    \begin{enumerate}
        \item \(A_i \neq \emptyset\) for all \(i \in \Lambda\)
        \item \(A_i \cap A_j = \emptyset\) for any \(i \neq j\)
        \item \(\bigcup_{i \in \Lambda} A_i = S\)
    \end{enumerate}
\end{definition}

\begin{nexample}
    \phantom{wow}

    \begin{itemize}
        \item \(S = \mathbb{Z}\), a valid partition would be \(P = \left\{\mathbb{Z}^-, \mathbb{Z}^+, \left\{0\right\}\right\}\).
        \item For the same \(S\), \(\left\{S\right\}\) is also a partition as well (lazy)
        \item For the same \(S\), let \(n = 3\), define
            \[
            \begin{aligned}
                A_0 &= \left\{0, \pm 3, \pm 6, \pm 9, \ldots\right\} \\
                A_1 &= \left\{\ldots, -5, -2, 1, 4, 7, 10, \ldots\right\} \\
                A_2 &= \left\{\ldots, -4, -1, 2, 5, 8, \ldots\right\}
            \end{aligned}
            \]
            and the partition \(P = \left\{A_0, A_1, A_2\right\}\) where each set in the partition is constructed with modulo 3
    \end{itemize}
\end{nexample}

\begin{definition}[Equivalence Class]
    Let \(\sim_{R}\) be an equivalence relation on \(S\) and let \(a \in S\). An equivalence class of \(A\), denoted \([a]_R\) or \([a]\), is the set \(\left\{x \in S : x \sim a\right\}\). That is, it is a set of \(x\)'s such that \((a, x) \in R\) or that \(a\) and \(x\) are equivalent.
\end{definition}

\begin{theorem}[Equivalence Classes Partition]
    Following are equivalent.

    \begin{enumerate}
        \item The equivalence classes of an equivalence relation on a set \(S\) is a partition of \(S\).
        \item For any partition \(P\) of \(S\), there is an equivalence relation on \(S\) whose equivalent classes are the the elements of \(P\).
    \end{enumerate}
\end{theorem}

\begin{proof}
    We will prove the two conditions in the theorem separately.

    \textit{Part 1.} Let \(\sim_R\) be a an equivalence relation on \(S\). Define a collection of subsets of \(S\) as follows:
    \[
        P = \left\{[a] : a \in S\right\}
    \]
    We will show that \(P\) is a partition.
    \begin{enumerate}
        \item We will show that the equivalent class is non-empty. Let \(a \in S\). Since \(a \sim a\), we have \(a \in [a]\). That is, \([a] \neq \emptyset\).
        \item We will show that \([a]\) and \([b]\) are equivalent classes such that \([a] \neq [b]\). That is, we want to show that \([a] \cap [b] \neq \emptyset\).

            % Without loss of generality, assume that there is \(b_1 \in [b] \setminus [a]\).
            Assume to the contrary that \([a] \cap [b] \neq \emptyset\). Then, there is some \(c \in [a] \cap [b]\). If \(a_1 \in [a]\) and \(b_1 \in [b]\), then we have that \(a_1 \sim c\) and \(c \sim b_1\). Therefore, \(a_1 \sim b_1\) by transtivity of being equivalence relations. Thus, \([a] = [b]\). That is, \(a_1 \in [b]\) and \(b_1 \in [a]\). This is a contradiction.

        \item Since \(S\) is the index, we already have that \(\bigcup_{a \in S} [a] = S\)
    \end{enumerate}

    \textit{Sketch for Part 2.} Let \(P\) be a partition of a set \(S\), say \(P = \left\{A_i : A_i \subseteq S, i \in \Lambda\right\}\). Define a relation \(R\) on S by: For any \(a, b \in S\), \((a, b) \in R\) if there is \(i \in \Lambda\) such that \(a, b \in A_i\). Using this, we can show that this is an equivalence relation.
\end{proof}
