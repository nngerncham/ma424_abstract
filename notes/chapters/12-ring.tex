\chapter{Introduction to Rings}

\begin{definition}[Rings]
    A ring \(R\) with two binary operations: addition \(+\) and multiplication \(\cdot\) such that for all \(a, b, c \in R\):
    \begin{enumerate}
        \item \(a + b = b + a\)
        \item \((a + b) + c = a + (b + c)\)
        \item There is an additive identity \(0\) such that \(a + 0 = a\) for all \(a \in R\)
        \item For all \(a \in R\), there is \(-a \in R\) such that \(a + (-a) = 0\)
        \item \((ab)c = a(bc)\)
        \item \(a(b+c) = ab + ac\) and \((b+c)a = ba + ca\)
    \end{enumerate}
\end{definition}

\begin{remark}
    If a set \(A\) has more than 2 operations, then we call it \textit{universal algebra}.
\end{remark}

\begin{remark}
    If \(R\) has a multiplicative identity, say 1, then \(R\) is called \textit{a ring with unity}.
\end{remark}

\begin{theorem}
    Let \(a, b, c \in R\). Then,
    \begin{enumerate}
        \item \(a0 = 0a = 0\)
        \item \(a(-b) = (-a)b = -(ab)\)
        \item \((-a)(-b) = ab\)
        \item \(a(b-c) = ab - ac\) and \((b-c)a = ba - ca\)
        \item If a ring has unity, then \(-1)a = -a\)
        \item If a ring has unity, then \((-1)(-1) = 1\)
    \end{enumerate}
\end{theorem}

\begin{theorem}
    If a ring has a unity, then it is unique. Similarly, if an element in a ring has a multiplicative inverse, then it is unique.
\end{theorem}

\begin{definition}[Subring]
    A set \(S \subseteq R\) of a ring \(R\) is called a \textit{subring} if it itself is a ring with the same operations with \(R\).
\end{definition}
