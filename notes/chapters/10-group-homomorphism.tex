\chapter{Group Homomorphisms}

Remember that
\begin{itemize}
    \item Homo means like
    \item Morphism refers to change form
\end{itemize}

\section{Definitions and Examples}

\begin{definition}[Homomorphism]
    A \textit{homomorphism} \(\phi : G \to \overline{G}\) is a function from \(G\) to \(\overline{G}\) that is operation-preserving. That is, \(\phi(ab) = \phi(a)\phi(b)\) for all \(a, b \in G\).
\end{definition}

Notice that with homomorphism, it does not require \(\phi\) to be 1-to-1 or onto. You can think of this as a relaxation of isomorphism.

\begin{nexample}
    Consider \(\phi: \mathbb{Z} \to \mathbb{Z}_5\). We would have
    \[
    \begin{aligned}
        \phi(x) &= x \mod 5
    \end{aligned}
    \]
    We can see that \(\phi(ab) = \phi(a)\phi(b)\) for any \(a, b \in \mathbb{Z}\). Therefore, we have that \(\phi\) is a homomorphism but no an isomorphism.
\end{nexample}

\begin{remark}
    If \(\phi : G \to \overline{G}\) is an isomorphism, then \(\phi\) is also a homomorphism.
\end{remark}

\begin{nexample}
    Recall that \(\mathbb{R}[x] = \left\{f : \mathbb{R} \to \mathbb{R} : f(x) = a_n x^n + \cdots + a_0 x^0\right\}\). Consider the map \(\phi : \mathbb{R}[x] \to \mathbb{R}[x]\) by
    \[
        \phi(f(x)) = f'(x)
    \]
    Then, we can see that
    \[
    \begin{aligned}
        \phi(f(x) g(x)) &= [f(x) + g(x)]' \\
                        &= f'(x) + g'(x) \\
                        &= \phi(f(x)) + \phi(g(x))
    \end{aligned}
    \]
    Thus, \(\phi\) is a homomorphism. However, \(\phi\) is not an isomorphism.

    Choose \(f_1(x) = 2\) and \(f_2(x) = 5\). We can see that \(f_1, f_2 \in \mathbb{R}[x]\). Then,
    \[
    \begin{aligned}
        \phi(f_1(x)) &= 2x \\
                     &= \phi(f_2(x)) \\
    \end{aligned}
    \]
    However, \(f_1(x) \neq f_2(x)\). So, \(\phi\) is not 1-to-1 and is not an isomorphism.
\end{nexample}

\begin{definition}[Kernel]
    Let \(\phi : G \to \overline{G}\) be a homomorphism and \(\overline{e}\) is the identity of \(\overline{G}\). Then, the \textit{kernel} of \(\phi\) is the set
    \[
        \ker(\phi) = \left\{x \in G : \phi(x) = \overline{e}\right\}
    \]
\end{definition}

\begin{theorem}
    Let \(\phi : G \to \overline{G}\) be a homomorphism and \(g \in G\). Then, we have the following properties:
    \begin{enumerate}
        \item \(\phi(e) = \overline{e}\) where \(e\) and \(\overline{e}\) are identities of \(G\) and \(\overline{G}\), respectively
        \item \(\phi(g^n) = \phi(g)^n\) for any \(n \in \mathbb{Z}\)
        \item If \(|g| < \infty\), then \(|\phi(g)|\) divides \(|g|\)
        \item \(\ker\phi\) is a subgroup of \(G\)
    \end{enumerate}
\end{theorem}

\begin{proof}
    \phantom{gg}

    \textit{Part 1.} Let \(e \in G\) be the identity of \(G\). Then, we will have that
    \[
    \begin{aligned}
        \phi(e) &= \phi(e e) \\
        \phi(e) &= \phi(e) \phi(e) \\
        \phi(e)\overline{e} &= \phi(e)\phi(e) \\
        \overline{e} &= \phi(e)
    \end{aligned}
    \]
    \(\phi(e) = \phi(e e) = \phi(e)\phi(e)\).

    \textit{Part 2.} Consider
    \[
    \begin{aligned}
        \phi(g^n) &= \phi(\underbrace{g g \cdots g}_\text{\(n\) times}) \\
                  &= \underbrace{\phi(g) \phi(g) \cdots \phi(g)}_\text{\(n\) times} \\
                  &= \phi(g)^n
    \end{aligned}
    \]

    \textit{Part 3.} Assume that \(|g| = k < \infty\). Then, we will have that
    \[
    \begin{aligned}
        \overline{e} &= \phi(e) \\
                     &= \phi(g^k) \\
                     &= \phi(g)^k
    \end{aligned}
    \]
    Thus, \(|\phi(g)|\) must divide \(k\).

    \textit{Part 4.} First, we already have that \(\ker\phi \subseteq G\). We will now use the 2-step Subgroup Test to show that \(\ker\phi \leq G\).
    \begin{enumerate}
        \item Let \(a, b \in \ker\phi\). Thus, \(\phi(a) = \overline{e} = \phi(b)\). Then, 
            \[
            \begin{aligned}
                \phi(ab) &= \phi(a) \phi(b) \\
                         &= \overline{e} \overline{e} \\
                         &= \overline{e}
            \end{aligned}
            \]
        \item Let \(a \in \ker\phi\). Therefore,
            \[
            \begin{aligned}
                \phi(a) &= \overline{e} \\
                (\phi(a))^{-1} &= (\overline{e})^{-1} \\
                (\phi(a))^{-1} &= \phi(a^{-1})
            \end{aligned}
            \]
            Thus, \(\phi(a^{-1}) = \overline{e}\) and \(a^{-1} \in \ker\phi\).
    \end{enumerate}
        Therefore, by the 2-step subgroup test, we can conclude that \(\ker\phi \leq G\).
\end{proof}

\begin{definition}[]
    Let \(g \in G\). Define
    \[
        g\ker\phi = \left\{x \in G : \phi(x) = \phi(g)\right\}
    \]
\end{definition}

Notice that letting \(a \in \ker\phi\). Then,
\[
\begin{aligned}
    \phi(ga) &= \phi(g)\phi(a) \\
             &= \phi(g)\overline{e} 
\end{aligned}
\]
That is, you can consider \(g\ker\phi\) to be the left coset of the kernel as well.

\begin{theorem}[Subgroups under Homomorphism]
    \phantom{gg}

    \begin{enumerate}
        \item \(\phi(a) = \phi(b)\) if and only if \(a\ker\phi = b\ker\phi\)
        \item If \(\phi(g) = g'\), then \(\phi^{-1}(g') = \left\{x \in G : \phi(x) = g'\right\} = g\ker\phi\)
    \end{enumerate}
\end{theorem}

\begin{proof}
    \phantom{g fucking g}

    \textit{Part 1.} We will first prove the forward case. Assume that \(\phi(a) = \phi(b)\) and let \(x \in a\ker\phi\). Thus, \(\phi(x) = \phi(a) = \phi(b)\). Hence, it follows that \(x \in b\ker\phi\). That is, \(a\ker\phi \subseteq b\ker\phi\). Similarly, we can prove that \(b\ker\phi \subseteq a\ker\phi\) using the same maneuver. Thus, \(a\ker\phi = b\ker\phi\).

    Now, we will prove the converse case. Assume that \(a\ker\phi = b\ker\phi\). Note that \(a \in a\ker\phi\) and \(b \in b\ker\phi\). Since \(a\ker\phi = b\ker\phi\), we also have that \(a \in b\ker\phi\) and \(b \in a\ker\phi\). Thus, \(\phi(a) = \phi(b)\).

    \textit{Part 2.} Assume that \(\phi(g) = g'\). We will show that \(\phi^{-1}(g') = g\ker\phi\). Let \(x \in \phi^{-1}(g') = \left\{a \in G : \phi(a) = g' = \phi(g)\right\}\). Thus, \(\phi(x) = g' = \phi(g)\) and we have that \(x \in g\ker\phi\).

    Now, we will show that \(y \in g\ker\phi\). Then, \(\phi(y) = \phi(g) = g'\) and we have that \(y \in \phi^{-1}(g')\).
\end{proof}

\begin{theorem}[Subgroups under Homomorphism]
    Let \(\phi : G \to \overline{G}\) be a group with homomorphism such that \(H \leq G\). Then, we have the following properties:
    \begin{enumerate}
        \item \(\phi(H) \leq \overline{G}\)
        \item If \(H\) is cyclic, then \(\phi(H)\) is cyclic as well
        \item If \(H\) is Abelian, then \(\phi(H)\) is Abelian as well
        \item If \(H\) is normal, then \(\phi(H)\) is normal as well
        \item If \(|\ker\phi| = n\), then \(\phi\) is an \(n\)-to-1 mapping
        \item \(|\phi(H)|\) divides \(|H|\)
        \item If \(\overline{K} \leq G\), then \(\phi^{-1}(K) \leq G\)
        \item If \(\overline{K} \unlhd G\), then \(\phi^{-1}(\overline{K}) \unlhd \overline{G}\)
        \item If \(\phi\) is onto and \(\ker\phi = \left\{e\right\}\), then \(\phi\) is an isomorphism from \(G\) to \(\overline{G}\)
    \end{enumerate}
\end{theorem}

\begin{proof}
    \phantom{gg}

    \textit{Part 1.} We will show that \(\phi(H) \leq \overline{G}\) using the 2-step subgroup test. First, let \(x, y \in \phi(H)\). Thus, there are \(a, b \in H\) such that \(x = \phi(a)\) and \(y = \phi(b)\). Therefore, \(xy = \phi(a)\phi(b) = \phi(ab) \in \phi(H)\).

    Now, let \(x \in \phi(H)\). We will show that \(x^{-1} \in \phi(H)\). Thus, there is \(a \in H\) such that \(x^{-1} = \phi(a)\). Hence, by the 2-step subgroup test, \(\phi(H) \leq G\).

    \textit{Part 2.} Let \(H = \langle g \rangle\) and let \(x \in \phi(H)\). There is \(a = g^i \in H\) such that \(\phi(a) = x\) for some \(i \in \mathbb{Z}\). Thus, \(x = \phi(a) = \phi(g^i) = (\phi(g))^i\). We have that \(\phi(H) = \langle \phi(g) \rangle\).

    \textit{Part 4.} We will show that if \(H \unlhd G\), then \(\phi(H) \unlhd \overline{G}\). Assume that \(H \unlhd G\). We will show that for any \(x \in \overline{G}\), \(x\phi(H)x^{-1} \subseteq \phi(H)\). 

    Let \(x \in \overline{G}\) and \(g' \in \phi(H)\). Therefore, there are \(a \in G\) and \(g \in H\) such that \(x = \phi(a)\) and \(g' = \phi(g)\). Then,
    \[
    \begin{aligned}
        xg'x^{-1} &= \phi(a) \phi(g) \phi(a^{-1}) \\
                  &= \phi(aga^{-1})
    \end{aligned}
    \]
    Since \(H \unlhd G\), \(aga^{-1} \in H\). Therefore, \(xg'x^{-1} \in \phi(H)\). We have \(x\phi(Hx^{-1) \subseteq \phi(H)})\) for any \(x \in \phi(G)\). Thus, \(\phi(H) \unlhd \phi(G)\).
\end{proof}
