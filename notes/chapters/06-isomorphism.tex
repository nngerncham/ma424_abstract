\chapter{Isomorphism}

\begin{nexample}
    Consider \((\mathbb{Z}_4, +) = \left\{[0], [1], [2], [3]\right\}\). Consider also the \((U(5), {}\cdot{}) = \left\{[1], [2], [3], [4]\right\}\).

    In \(\mathbb{Z}_4\), we can see that the elements are 
    \[
    \begin{aligned}
        & [1] \\
        & [2] = [1] + [1] \\
        & [3] = [1] + [1] + [1] \\
        & [0] = [4] = [1] + [1] + [1] + [1]
    \end{aligned}
    \]

    Similarly, we have the following in \(U(5)\):
    \[
    \begin{aligned}
        & [2] \\
        & [4] = [2] \cdot [2] \\
        & [3] = [8] = [2] \cdot [2] \cdot [2] \\
        & [1] = [16] = [2] \cdot [2] \cdot [2] \cdot [2]
    \end{aligned}
    \]
    We can see that \(\mathbb{Z}_4 = \langle [1] \rangle\) and \(U(5) = \langle [2] \rangle\). We can also see that \((\mathbb{Z}_4, +)\) is \textit{structurally similar} to \(U(5)\).
\end{nexample}

\begin{definition}[Isomorphism]
    Let \((G, {}\cdot{})\) and \((\overline{G}, {}*{})\) be groups. An \textit{isomorphism} from \(G\) to \(\overline{G}\) is a bijective function \(\phi : G \overunderset{\text{1-to-1}}{\text{onto}}{\to} \overline{G}\) such that for \(a, b \in G\), \(\phi(a \cdot b) = \phi(a) * \phi(b)\).
\end{definition}

\begin{definition}[Isomorphic]
    Groups \(G\) and \(\overline{G}\) are \textit{isomorphic} if there is an isomorphism between them.
\end{definition}

\begin{nexample}
    Consider \(G = (\mathbb{R}, {}+{})\) and \(\overline{G} = (\mathbb{R}^+, {}\cdot{})\). We will show that they are isomorphic. First, define the isomorphism \(\phi: G \to \overline{G}\) to be
    \[
        \phi(x) = e^x
    \]
    for \(x \in \mathbb{R}\).

    Now, we will show that the function is 1-to-1. Since the exponential function is increasing on \(\mathbb{R}\), \(\phi\) is 1-to-1. Next, we will show that the function is onto. Let \(y \in \mathbb{R}^+\). Choose \(x = \ln y\). Then, we have that \(e^x = e^{\ln y} = y\).

    Finally, we will show that \(\phi\) is operation-preserving. Let \(a, b \in \mathbb{R}\). Consider
    \[
    \begin{aligned}
        \phi(a+b) &= e^{a+b} \\
                  &= e^a \cdot e^b \\
                  &= \phi(a) \cdot \phi(b)
    \end{aligned}
    \]
\end{nexample}

\begin{nexample}
    Consider \(G = S_3\). Recall that 
    \[
        S_3 = \left\{\varepsilon, (1\ 2), (1\ 3), (2\ 3) (1\ 2\ 3), (1\ 3\ 2)\right\}
    \]
    Then, define the function \(\phi: S_3 \to S_3\) by
    \[
        \ \forall \ \alpha \in S_3, \phi(\alpha) = (1\ 2\ 3)\alpha
    \]

    To show that \(\phi\) is 1-to-1, we will assume that \(\phi(\alpha_1) = \phi(\alpha_2)\) and show that \(\alpha_1 = \alpha_2\). Consider
    \[
    \begin{aligned}
        \phi(\alpha_1) &= \phi(\alpha_2) \\
        (1\ 2\ 3)\alpha_1 &= (1\ 2\ 3)\alpha_2 \\
        (1\ 2\ 3)^{-1}(1\ 2\ 3) \alpha_1 &= (1\ 2\ 3)^{-1}(1\ 2\ 3) \alpha_2 \\
        \alpha_1 &= \alpha_2
    \end{aligned}
    \]

    To show that \(\phi\) is onto, let \(\beta \in S_3\). Choose \(\alpha_0 = (1\ 2\ 3)^{-1}\beta\). Then, \(\phi(\alpha_0) = (1\ 2\ 3)[(1\ 2\ 3)^{-1}\beta] = \beta\).

    To show that \(\phi\) is operation preserving, let \(\tau_1, \tau_2 \in S_3\). Consider 
    \[
    \begin{aligned}
        \phi(\tau_1 \tau_2) &= (1\ 2\ 3)(\tau_1 \tau_2) \\
                            &= 
    \end{aligned}
    \]
\end{nexample}

\begin{theorem}[Cayley's Theorem]
    Every group is isomorphic to a subgroup of permutations. That is,
    \[
        G \cong \overline{G} \text{ where } \overline{G} \leq S_G
    \]
\end{theorem}

\begin{proof}
    First, define \(\phi: G \to \overline{G} \leq S_G\) where \(\overline{G} = \left\{\phi_g \in S_G : \phi_g(x) = gx, x \in G\right\}\) as follows: For any \(g \in G\), \(\phi(g) = \phi_g\) where \(\phi_g(x) = gx\) for some \(x \in G\). Claim that \(\phi\) is an isomorphism.

    First, we will show that it is 1-to-1. Assume that \(g_1, g_2 \in G\) be such that \(\phi(g_1) = \phi(g_2)\). Thus, \(\phi_{g_1} = \phi_{g_2}\) as functions. Choose \(x = e \in G\). Consider
    \[
    \begin{aligned}
        \phi_{g_1}(e) &= \phi_{g_2}(e) \\
        g_1 e &= g_2 e \\
        g_1 &= g_2
    \end{aligned}
    \]

    Now, we will show that \(\phi\) is onto. Let \(T \in \overline{G} \subseteq S_G\). Thus, there is \(g_0 \in G\) such that \(T(x) = g_0 x\) for any \(x \in G\). Choose \(g = g_0\). Then, we will have that \(\phi(g) = T\).

    Finally, we will show that \(\phi\) preserves operation and \(\overline{G} < S_G\). Namely, we will show that for \(g_1, g_2 \in G\), \(\phi(g_1 g_2) = \phi(g_1) \circ \phi(g_2)\). Let \(x \in G\). Consider
    \[
    \begin{aligned}
        \phi(g_1 g_2)(x) &= \phi_{g_1 g_2}(x) \\
                         &= (g_1 g_2)x \\
                         &= g_1(g_2 x) \\
                         &= g_1(\phi_{g_2}(x)) \\
                         &= (\phi_{g_1} \circ \phi_{g_2}) (x)
    \end{aligned}
    \]
    Thus, \(\phi(g_1 g_2) = \phi_{g_1 g_2}\). Hence, \(\phi\) is an isomorphism.
\end{proof}
