\chapter{Finite Groups and Subgroups}

\begin{definition}[Order of a Group]
    The number of elements of a group \(G\) is called its \textit{order}. We write \(|G|\) to represent the order of \(G\). If \(G\) is not finite, then we write \(|G| = \infty\).
\end{definition}

\begin{definition}[Order of an Element]
    The order of \(g \in G\) is the smallest \(n \in \mathbb{N}\) such that \(g^n = e\). Then, we write \(|g| = n\). If no such integer exists, then we say \(|g| = \infty\).
\end{definition}

\begin{nexample}
    Consider \(U(6) = \left\{[1], [5]\right\}\). Notice that \(|U(6)| = 2\) since there are two elements. Observe that 
\end{nexample}

\begin{definition}[Subgroup]
    Let \((G, {}*{})\) be a group. If \(H \subseteq G\) and \((H, {}*{})\) is also a group, then \((H, {}*{})\) is called a subgroup of \(G\). We denote this as \(H \leq G\). Similarly, if \(H\) is a subgroup of \(G\) but is not \(G\), then we denote this by \(H < G\) and we call it a \textit{proper subgroup}. If \(H = \left\{e\right\}\), then it is called a trivial subgroup. Otherwise, it is a non-trivial subgroup.
\end{definition}

\begin{nexample}
    Consider \(G = \left\{A \in M_{2 \times 2}(\mathbb{R}) : \det(A) \neq 0\right\}\). We know that \((G, {}\cdot{})\) is a group. Now, consider \(H_1 = \left\{A \in M_{2 \times 2}(\mathbb{Z}) : \det(A) \neq 0\right\}\). Is \(H \leq G\)?
\end{nexample}
