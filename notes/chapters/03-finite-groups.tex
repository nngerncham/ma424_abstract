\chapter{Finite Groups and Subgroups}

\section{Subgroups}

\begin{definition}[Order of a Group]
    The number of elements of a group \(G\) is called its \textit{order}. We write \(|G|\) to represent the order of \(G\). If \(G\) is not finite, then we write \(|G| = \infty\).
\end{definition}

\begin{definition}[Order of an Element]
    The order of \(g \in G\) is the smallest \(n \in \mathbb{N}\) such that \(g^n = e\). Then, we write \(|g| = n\). If no such integer exists, then we say \(|g| = \infty\).
\end{definition}

\begin{nexample}
    Consider \(U(6) = \left\{[1], [5]\right\}\). Notice that \(|U(6)| = 2\) since there are two elements. Observe that 
\end{nexample}

\begin{definition}[Subgroup]
    Let \((G, {}*{})\) be a group. If \(H \subseteq G\) and \((H, {}*{})\) is also a group, then \((H, {}*{})\) is called a subgroup of \(G\). We denote this as \(H \leq G\). Similarly, if \(H\) is a subgroup of \(G\) but is not \(G\), then we denote this by \(H < G\) and we call it a \textit{proper subgroup}. If \(H = \left\{e\right\}\), then it is called a trivial subgroup. Otherwise, it is a non-trivial subgroup.
\end{definition}

\begin{nexample}
    Consider \(G = \left\{A \in M_{2 \times 2}(\mathbb{R}) : \det(A) \neq 0\right\}\). We know that \((G, {}\cdot{})\) is a group. Now, consider \(H_1 = \left\{A \in M_{2 \times 2}(\mathbb{Z}) : \det(A) \neq 0\right\}\). Is \(H \leq G\)? The answer is \textit{no}.

    Consider
    \[
        A = \begin{bmatrix}
            1 & 0 \\
            0 & 2
        \end{bmatrix}
    \]
    Observe that \(\det(A) = 2\). Then, we would have that
    \[
        A^{-1} = \begin{bmatrix}
            1 & 0 \\
            0 & \frac{1}{2}
        \end{bmatrix}
    \]
    However, it is clear that \(A^{-1} \notin H_1\) since it has a non-integer entry.
\end{nexample}

\begin{nexample}
    Consider \(H_2 = \left\{A \in M_{2 \times 2}(\mathbb{Z}) : \det(A) = 1\right\}\). Observe that \(H_2 \subseteq G := (M_{2 \times 2}(\mathbb{R}), {}\cdot{})\). Is this a subgroup?

    First, let us check the closure property. Let \(A, B \in H_2\). Notice that \(\det(A \cdot B) = \det(A) \cdot \det(B) = 1\). Thus, \(AB \in H_2\).

    Next, let us check for associativity. Since \(H_2 \subseteq G\), it has the associativity under the same operation.

    Now, for identity. The identity for \(G\) is the identity matrix. It is clear that \(\det(I) = 1\), meaning that \(I \in H_2\).

    Finally, for inverse. Consider
    \[
        A = \begin{bmatrix}
            a & b \\
            c & d
        \end{bmatrix} \in H_2
    \]
    Thus, \(a, b, c, d \in \mathbb{Z}\) with \(ad - bc = 1\). Recall that the inverse of a \(2 \times 2\) matrix is
    \[
        A^{-1} = \frac{1}{ad - bc}\begin{bmatrix}
            d & -b \\
            -c & a
        \end{bmatrix}
    \]
    Hence, it follows that
    \[
    \begin{aligned}
        A^{-1} &= \frac{1}{ad - bc}\begin{bmatrix}
                    d & -b \\
                    -c & a
               \end{bmatrix} \\
               &= \frac{1}{1}\begin{bmatrix}
                   d & -b \\
                   -c & a
               \end{bmatrix} \\
               &= \begin{bmatrix}
                   d & -b \\
                   -c & a
               \end{bmatrix}
    \end{aligned}
    \]
    Now, notice that \(\det(A^{-1}) = da - (-b)(-c) = ad - bc = 1\). Thus, we can conclude that \(A^{-1} \in H_2\) and that it is a subgroup of \(G\). This subgroup is called the \textit{Special Linear Group over \(\mathbb{Z}\)}, denoted as \(SL_2(\mathbb{Z})\).
\end{nexample}

\section{Subgroup Tests}

\begin{theorem}[One-step Subgroup Test]
    Let \(G\) be a group and \(\emptyset \neq H \subseteq G\). Assume that for all \(a, b \in H\), we have that \(ab^{-1} \in H\). Then, we can conclude that \(H \leq G\).
\end{theorem}

\begin{proof}
    \phantom{Part 0}

    \underline{\textit{Part 1: Associativity.}} Since \(H \subseteq G\), then the operation is associative in \(H\). 

    \underline{\textit{Part 2: Identity.}} Since \(H \neq \emptyset\), there is \(x \in H\). Let \(a = x\) and \(b = x\). By the assumption, \(ab^{-1} \in H\). That is, \(xx^{-1} = e \in H\). Thus, the identity exists in \(H\).

    \underline{\textit{Part 3: Inverse.}} Let \(y \in H\). Choose \(a = e \in H\) and \(b = y \in H\). Then, by the assumption, \(ab^{-1} = y^{-1}\) must be in \(H\) as well.

    \underline{\textit{Part 4: Closure}}. Let \(x_1, x_2 \in H\). We will show that \(x_1x_2 \in H\). From part 3, we already have that \(x_2^{-1} \in H\). Choose \(a = x_1\) and \(b = x_2^{-1}\). By the assumption, we have
    \[
    \begin{aligned}
        ab^{-1} &= x_1(x_2^{-1})^{-1} \\
                &= x_1x_2 \in H
    \end{aligned}
    \]
    Thus, we have that \(x_1x_2 \in H\).
\end{proof}

\begin{nexample}
    Let \(G\) be an Abelian group. Define \(H = \left\{x \in G : x^2 = e\right\}\). Claim that \(H \leq G\).

    \begin{proof}
        Note that \(H \neq \emptyset\) since \(e^2 = e \in H\). Assume that \(a, b \in H\). That means that \(a^2 = e = b^2\). We will now show that \(ab^{-1} \in H\). That is, \(ab^{-1} = e\).

        Consider
        \[
        \begin{aligned}
            (ab^{-1})^2 &= (ab^{-1})(ab^{-1})  \\
                        &= (aa)(b^{-1}b^{-1}) \\
                        &= a^2(b^{-1})^2 \\
                        &= a^2(b^{2})^{-1} \\
                        &= ee^{-1} \\
                        &= e
        \end{aligned}
        \]

        Thus, we have that \(ab^{-1} = e\). Therefore, by the one-step subgroup test, we have that \(H \leq G\).
    \end{proof}
\end{nexample}

\begin{theorem}[Two-step Subgroup Test]
    Let \(G\) be a group and \(\emptyset \neq H \subseteq G\). Then, \(H \leq G\) if the following conditions satisfy:
    \begin{enumerate}
        \item For \(a, b \in H\), then \(ab \in H\)
        \item For \(a \in H\), then \(a^{-1} \in H\)
    \end{enumerate}
\end{theorem}

\begin{proof}
    It follows from the one-step subgroup test.
\end{proof}

\begin{nexample}
    Let \(G\) be an Abelian group. Define \(H = \left\{a \in G : |a| < \infty\right\}\). We claim that \(H \leq G\).

    \begin{proof}
        Since \(|e| = 1 < \infty\), we have that \(e \in H\). Thus, \(H\) is not empty. Let \(a, b \in H\). We have that \(|a| = m\) and \(|b| = n\) for \(m, n \in \mathbb{N}\).

        \underline{\textit{Part 1: Closure.}} We will show that \(ab \in H\). Consider
        \[
        \begin{aligned}
            (ab)^{mn} &= \underbrace{(ab)(ab)\cdot(ab)}_\text{\(mn\) times} \\
                      &= a^{mn}b^{mn} \\
                      &= (a^m)^n(b^n)^m \\
                      &= e^ne^m \\
                      &= ee = e
        \end{aligned}
        \]
        Thus, we have that \(|ab| < \infty\).

        \underline{\textit{Part 2: Inverse.}} Let \(c \in H\). That is, \(|c| = k < \infty\). Thus, \(c^k = e\). Consider
        \[
        \begin{aligned}
            (c^{-1})^k &= (c^k)^{-1} \\
                       &= e^{-1} = e
        \end{aligned}
        \]
        Thus, \(|c^{-1}| < \infty\) and we can conclude that \(c^{-1} \in H\).

        Therefore, by the two-step subgroup test, \(H\) is a subgroup of \(G\).
    \end{proof}

    \begin{remark}
        For any \(x \in G\), \(|x| = |x^{-1}|\).
    \end{remark}
\end{nexample}

\begin{nexample}
    Let \(G\) be an Abelian group. Let \(H, K \leq G\). Define \(HK = \left\{hk \in G : h \in H, k \in K\right\}\). We claim that \(HK \leq G\). The proof can be done using the 2-step subgroup test.
\end{nexample}

\begin{theorem}[Finite Subgroup Test]
    Let \(G\) be a group and \(\emptyset \neq H \subseteq G\). If \(H\) satisfies the following conditions:
    \begin{enumerate}
        \item \(H\) is finite, and
        \item For every \(a, b \in H\), \(ab \in H\)
    \end{enumerate}
    Then, \(H\) is a subgroup of \(G\). Note that if \(G\) is already finite, then the first condition is trivially true.
\end{theorem}

\begin{proof}
    \phantom{help}

    \underline{\textit{Part 1: Closure.}} The closure property can be directly derived from the second condition.

    \underline{\textit{Part 2: Inverse.}} Let \(a \in H\). Consider \(S = \left\{a^k \in H : k \in \mathbb{N}\right\}\). We have that \(H \subseteq G\). Since \(H\) is finite, so is \(S\). Thus, there are \(a^m = a^n\) for some \(m, n \in \mathbb{N}\) and \(m < n\). Then, we have that
    \[
    \begin{aligned}
        \underbrace{a \cdots a}_{m \in \mathbb{N}}
            &= \overbrace{\underbrace{a \cdots a}_{m \in \mathbb{N}}\underbrace{a \dots a}_{n-m \in \mathbb{N}}}^{n \in \mathbb{N}} \\
        a^m &= \underbrace{a^m a^{n-m}}_{a^n} \\
        e &= a^{n-m}
    \end{aligned}
    \]
    If \(n-m = 1\), then \(e = a\). Thus, \(a^{-1} = e \in H\).

    Otherwise, if \(n-m > 1\), then
    \[
    \begin{aligned}
        e &= a^{n-m} \\
          &= a\underbrace{a^{n-m-1 \geq 1}}_{a^{-1}}
    \end{aligned}
    \]
    Thus, we can choose \(b = a^{n-m-1}\) and we will have that \(ab = e = ba\). Both cases implies that \(a^{-1} \in H\).
\end{proof}

\begin{remark}
\end{remark}

\begin{theorem}
    Define the following notation:
    \[
        \langle a \rangle = \left\{a^n \in G : n \in \mathbb{Z}\right\}
    \]

    Let \(G\) be a group and \(a \in G\). Then, \(\langle a \rangle\) is an Abelian subgroup of \(G\).
\end{theorem}

\begin{proof}
    Left as an exercise. This can be done using the 1SST or 2SST.
\end{proof}

\section{Center of Groups}

\begin{definition}[Center of a Group]
    Let \(G\) be a group. Define
    \[
        Z(G) = \left\{a \in G : \ \forall \ a \in G, ax = xa\right\}
    \]
    and is called the \textit{center} of group \(G\).
\end{definition}

\begin{distraction}
    The \(Z\) comes from the word \textit{``zentrum''} in German which translates to the center.
\end{distraction}

\begin{theorem}\label{thm:subgroup-center}
    \[
        Z(G) \leq G
    \]
\end{theorem}

\begin{proof}
    \underline{\textit{Part 1: Identity.}} Let \(x \in G\). We have that \(ex = xe\). Thus, \(e \in Z(G)\) and \(Z(G) \neq \emptyset\).

    \underline{\textit{Part 2: Invertibility.}} Let \(a \in Z(G)\). We want to show that \(a^{-1} \in Z(G)\) as well. For any \(x \in G\), we have that \(ax = xa\). Consider
    \[
    \begin{aligned}
        a^{-1}(ax)a^{-1} &= a^{-1}(xa)a^{-1} \\
        (a^{-1}a)xa^{-1} &= a^{-1}x(aa^{-1}) \\
        xa^{-1} &= a^{-1}x
    \end{aligned}
    \]
    Thus, \(a^{-1} \in Z(G)\).

    By the 2-step subgroup test, we can conclude that \(Z(G) \leq G\).
\end{proof}

\begin{nexample}
    Let \(G\) be an Abelian group. Then, for every \(a \in G\) and \(x \in G\). Thus, \(Z(G) = G\).
\end{nexample}

\begin{nexample}
    Consider the group \(D_3 = \left\{V_1, V_2, V_3, R_1, R_2, e\right\}\). Observe that \(\left|D_3\right| = 6\). Notice also that
    \[
    \begin{aligned}
        R_1 V_1 &= V_2  &\qquad\text{and}\\
        V_1 R_1 &= V_3 \\
        R_1 V_1 &\neq V_1 R_1
    \end{aligned}
    \]
    Thus, \(G\) is not Abelian. In fact, for any rotation \(R\) and any reflection \(V\), \(RV \neq VR\). Hence, we claim that \(Z(G) = \left\{e\right\}\).

    \begin{proof}
        Let \(x \in G \setminus \left\{e\right\}\). Then, \(x\) is a notation or a reflection. Choose \(y\) to be the other type of symmetry, i.e., if \(x\) is a rotation, then \(y\) is a reflection and vice versa. Then, \(xy \neq yx\). Thus, \(Z(G) = \left\{ e\right\}\).
    \end{proof}
\end{nexample}

\begin{definition}[Centralizer of \(a\) in \(G\)]
    The \textit{centralizer} of \(a \in G\) is defined by
    \[
        C(a) = \left\{g \in G : ag = ga\right\}.
    \]
\end{definition}

\begin{nexample}
    Let \(G = D_3\) and \(a = R_1\). We will have that
    \[
    \begin{aligned}
        C(a) &= C(R_1) \\
             &= \left\{e, R_1, R_1^2\right\} \\
             &= \langle R_1 \rangle
    \end{aligned}
    \]

    Similarly, let \(b = V_2\). We will have that
    \[
    \begin{aligned}
        C(b) &= \left\{e, V_2\right\}
    \end{aligned}
    \]
    If we 
\end{nexample}

\begin{theorem}
    Let \(a \in G\). Then, \(C(a) \leq G\) as well.
\end{theorem}

\begin{proof}
    The proof is similar to Theorem \ref{thm:subgroup-center}.
\end{proof}
