\newpage

\section*{Chapter 9}

\begin{hwproblem}
{15}{
    What is the order of the element \(4U_5(105)\) in the factor group \(U(105) / U_5(105)\).
}

Recall that the group operation in factor groups applied on itself \(a\) times would be
\[
    (4U_5(105))^a = 4^a U_5(105) \\
\]


To find the order of \(4U_5(105)\), we can do the computation directly. Consider
\[
\begin{aligned}
    4U_5(105) &= 4\left\{1, 11, 16, 26, 31, 41, 46, 61, 71, 76, 86, 101\right\} \\
              &= \left\{4, 44, 64, 104, 19, 59, 79, 34, 74, 94, 29, 89\right\} \\
    (4U_5(105))^2 &= 16U_5(105) \\
                  &= \left\{1, 11, 16, 26, 31, 41, 46, 61, 71, 76, 86, 101\right\} \\
                  &= U_5(105) \\
\end{aligned}
\]

Therefore, the order of \(4U_5(105) \in U(105)/U_5(105)\) is 2.
\end{hwproblem}

\begin{hwproblem}
{27}{
    Let $G=U(16), H=\{1,15\}$, and $K=\{1,9\}$. Are $H$ and $K$ isomorphic? Are $G / H$ and $G / K$ isomorphic?
}

Claim that \(H \approx K\) by \(\phi : H \to K\) such that
\[
    \phi(x) = \begin{cases}
        1 & \text{If } x = 1 \in H \\
        9 & \text{If } x = 15 \in H
    \end{cases}
\]
By definition, we already have that \(\phi\) is bijective. Now, we will (brute force) show that \(\phi\) is operation-preserving. Notice that \(U(16)\) is Abelian, and so are \(H\) and \(K\).
\[
\begin{aligned}
    \phi(1 \cdot 15) &= \phi(1) \phi(15) \\
    \phi(15) &= 1 \cdot 9 \\
    9 &= 9\\
    \text{And, } \quad \phi(1 \cdot 1) &= \phi(1) \cdot \phi(1) \\
        1 &= 1\\
    \text{And, } \quad \phi(15 \cdot 15) &= \phi(15) \cdot \phi(15) \\
        \phi(1) &= 9 \cdot 9 \\
        1 &= 1
\end{aligned}
\]
Therefore, \(H \approx K\).

However, claim that \(G/H \napprox G/K\). Note that \(|G| = 8\) so \(|G/H| = 4 = |G/K|\) since \(|H| = |K| = 2\) and factor groups are essentially a group of disjoint cosets with equal order. Note also that the order of cosets of \(H, K\) have the same order as \(H, K\). Recall also that the operation for elements in factor groups are: For \(a, b \in G'\) and \(H' \leq G'\), \((aH')(bH') = abH'\).

First, let us consider the elements of \(G\)
\[
    G = \left\{1, 3, 5, 7, 9, 11, 13, 15\right\}
\]

Next, consider all cosets \(aK\) of \(K\) for \(a \in G\):
\[
\begin{aligned}
    1K &= K                     &\quad 9K &= K 
       &\quad 3K &= \left\{3, 11\right\}  &\quad 11K &= \left\{11, 3\right\} \\
    5K &= \left\{5, 13\right\}  &\quad 13K &= \left\{13, 5\right\} 
       &\quad 7K &= \left\{7, 15\right\}  &\quad 15K &= \left\{15, 7\right\}
\end{aligned}
\]
Observe that
\[
\begin{aligned}
    &(1K)^2 = 1K = K \quad &(9K)^2 = 1K = K \\
    &(3K)^2 = 9K = K \quad &(11K)^2 = 9K = K \\
    &(5K)^2 = 9K = K \quad &(13K)^2 = 9K = K \\
    &(7K)^2 = 1K = K \quad &(15K)^2 = 1K = K \\
\end{aligned}
\]
Namely, we can see that every element in \(G/K\) has order 2.

Now, consider the element \(3H \in G/H\).
\[
\begin{aligned}
    3H &= \left\{3, 13\right\} \\
    (3H)^2 &= 9H \\
           &= \left\{7, 9\right\} \\
    (3H)^3 &= 11H \\
           &= \left\{5, 11\right\} \\
    (3H)^4 &= 1H = H
\end{aligned}
\]
Here, we can see that \(|3H| = 4\). Hence, it follows that it is impossible for us to find a bijective and operation-preserving function \(\phi : G/H \to G/K\). Assume to the contrary that there is such function \(\phi\). Consider \(h = 3H \in G/H\) and \(k \in G/K\) such that \(\phi(h) = k\). Then,
\[
\begin{aligned}
    \phi(h^3) &= \phi(h)^3 \\
              &= k^3 \\
              &= kk^2 \\
              &= k = \phi(h)
\end{aligned}
\]
However, since \(h \neq h^3\), we have a contradiction. Therefore, \(G/H \napprox G/K\).
\end{hwproblem}

\begin{hwproblem}
{34}{
    In \(\mathbb{Z}\), let \(H = \langle 5 \rangle\) and \(K = \langle 7 \rangle\). Prove that \(\mathbb{Z} = HK\). Does \(\mathbb{Z} = H \times K\)?
}

First, note that
\[
\begin{aligned}
    \langle 5 \rangle &= \left\{..., -10, -5, 0, 5, 10, ...\right\} \\
                      &= \left\{5a : a \in \mathbb{Z}\right\} \\
    \langle 7 \rangle &= \left\{..., -14, -7, 0, 7, 14, ...\right\} \\
                      &= \left\{7b : b \in \mathbb{Z}\right\}
\end{aligned}
\]

Next, recall Bezout's Theorem: For any \(n, m \in \mathbb{Z}\), there always exist \(x, y \in \mathbb{Z}\) such that \(xn + ym = \gcd(n, m)\). Thus, we have that there are always \(x, y \in \mathbb{Z}\) such that \(5x + 7y = \gcd(5, 7) = 1\). Then, for any \(z \in \mathbb{Z}\), we would have that 
\[
    z(5x + 7y) = 5zx + 7zy = z
\]
Since \(zx, zy \in \mathbb{Z}\), we have that \(5zx \in \langle 5 \rangle\) and \(7zy \in \langle 7 \rangle\). Therefore, we can conclude that \(\mathbb{Z} = HK = \left\{5c + 7d : c, d \in \mathbb{Z}\right\} = \langle 5, 7 \rangle\).

However, \(\mathbb{Z} \neq H \times K\) since \(H \cap K \neq \left\{e\right\}\). As a counterexample, consider 35. We can see that \(35 = 7 \cdot 5\) and thus is present in both \(\langle 5 \rangle\) and \(\langle 7 \rangle\). Thus, \(H \cap K = \left\{e, 35, ...\right\} \neq \left\{e\right\}\) and therefore, \(\mathbb{Z} \neq H \times K\).
\end{hwproblem}

\begin{hwproblem}
{42}{
    An element is called a square if it can be expressed in the form $b^2$ for some $b$.

    Suppose that $G$ is an Abelian group and $H$ is a subgroup of $G$. If every element of $H$ is a square and every element of $G / H$ is a square, prove that every element of $G$ is a square. Does your proof remain valid when ``square'' is replaced by ``$n$th power,'' where $n$ is any integer?
}

\textbf{Disclaimer.} Assuming that when consider the square \(b^2 \in B\) where \(B\) is a group, we are referring to the case where \(b \in B\) as well. Otherwise, this proof might fall apart, namely for the Abelian part.

Recall that when considering the factor group \(G/H\), we have that for any \(x, y \in G\), \( (xH)(yH) = xyH \). Thus, a coset, say \(zH\), being a square means that \( zH = (wH)(wH) = w^2H \) for some \(w\).

Now, claim that
\[
    \bigcup_{A \in G/H} A = G
\]
That is, the union of every element, coset \(aH\) for some \(a \in G\), of the factor group \(G/H\) forms \(G\). Recall that \(G/H = \left\{aH : a \in G\right\}\). Thus, we have that \(ae = a \in aH\) for every \(a \in G\) since \(aH \in G/H\) for any \(a \in G\). Thus, taking union of all elements of \(G/H\) forms \(G\).

Let us now prove the main claim. It suffices to show that every element of each coset in \(G/H\) is a square to prove that every element of \(G\) is a square since it will just follow from the previous claim. Let \(g \in G\) and consider \(gH \in G/H\). Since every element of \(G/H\) is a square, then there is \(k \in G\) such that \(gH = (kH)^2 = k^2H\). Furthermore, we have that
\[
\begin{aligned}
    (kH)^2 &= k^2H \\
           &= \left\{k^2h : h \in H\right\}
\end{aligned}
\]
Since every element in \(H\) is a square as well, we have that there is \(h' \in H\) such that \(h = (h')^2\). Then, we have
\[
\begin{aligned}
    k^2H &= \left\{k^2h : h \in H\right\} \\
         &= \left\{k^2(h')^2 : (h')^2 = h \in H\right\} \\
\end{aligned}
\]
Since \(G\) is Abelian, then so must \(H\). Hence, it follows that
\[
\begin{aligned}
    k^2H &= \left\{k^2(h')^2 : (h')^2 = h \in H\right\} \\
         &= \left\{(kh')(kh') : (h')^2 = h \in H\right\} \\
         &= \left\{(kh')^2 : (h')^2 = h \in H\right\}
\end{aligned}
\]
Thus, we have that every element of each coset \(gH = k^2H \in G/H\) is a square. Therefore, we can conclude that every element of \(G\) is a square.
\end{hwproblem}

\begin{hwproblem}
{56}{
    Show that the intersection of two normal subgroups of $G$ is a normal subgroup of $G$. Generalize.
}

Let \(H, K \unlhd G\) and let \(L = H \cap K\). We will show that \(L \unlhd G\). Namely, we will show that for any \(l \in L\), we have that \(xlx^{-1} \in L\) for any \(x \in G\).

Consider \(l \in L\). Then, \(l \in H\) and \(l \in K\). Since \(H, K \unlhd G\), we have that \(xlx^{-1} \in H\) and \(xlx^{-1} \in K\) for any \(x \in G\). Hence, it follows that \(xlx^{-1} \in H \cap K = L\) for any \(x \in G\). Thus, we have that \(xLx^{-1} = \left\{xlx^{-1} : l \in L, x \in G\right\} \subseteq L\). Therefore, by the normal subgroup test, we can conclude that \(L \unlhd G\).

The generalization of this claim is
\[
    \text{Let \(H_1, H_2, ... \unlhd G\). Then, 
    \(H = \bigcap_{\lambda \in \Lambda} H_\lambda \unlhd G\) for some index set \(\Lambda\).}
\]
\end{hwproblem}
