\section*{Chapter 0}

\begin{hwproblem}
{2}{
    Determine
    \begin{enumerate}[label=\alph*.]
        \item $\operatorname{gcd}(2,10), \quad \operatorname{lcm}(2,10)$
        \item $\operatorname{gcd}(20,8), \quad \operatorname{lcm}(20,8)$
        \item $\operatorname{gcd}(12,40), \quad \operatorname{lcm}(12,40)$
        \item $\operatorname{gcd}(21,50), \quad \operatorname{lcm}(21,50)$
        \item $\operatorname{gcd}\left(p^2 q^2, p q^3\right), \quad \operatorname{lcm}\left(p^2 q^2, p q^3\right)$ where $p$ and $q$ are distinct primes
    \end{enumerate}
}

\begin{enumerate}[label=\alph*.]
    \item $\operatorname{gcd}(2,10) = 2, \quad \operatorname{lcm}(2,10) = 10$
    \item $\operatorname{gcd}(20,8) = 4, \quad \operatorname{lcm}(20,8) = 40$
    \item $\operatorname{gcd}(12,40) = 4, \quad \operatorname{lcm}(12,40) = 120$
    \item $\operatorname{gcd}(21,50) = 1, \quad \operatorname{lcm}(21,50) = 1050$
    \item $\operatorname{gcd}\left(p^2 q^2, p q^3\right) = pq^2, \quad \operatorname{lcm}\left(p^2 q^2, p q^3\right) = p^2 q^3$
\end{enumerate}
\end{hwproblem}

\begin{hwproblem}
{16}{
    Determine \(7^{1000} \mod 6\) and \(6^{1001} \mod 7\).
}

\begin{itemize}
    \item For \(7^{1000} \mod 6\), consider:
        \[
        \begin{aligned}
            7^{1000} \mod 6 &= \left(\underbrace{7 \cdot 7 \cdot \ldots \cdot 7}_{\text{1000 times}}\right) \mod 6 \\
                            &= \left(\underbrace{(7 \mod 6) \cdot (7 \mod 6) \cdot \ldots \cdot (7 \mod 6)}_{\text{1000 times}}\right) \mod 6 \\
                            &= \left(\underbrace{1 \cdot 1 \cdot \ldots 1 \cdot 1}_{\text{1000 times}}\right) \mod 6 \\
                            &= 1 \mod 6 = 1
        \end{aligned}
        \]

    \item For \(6^{1001} \mod 7\), consider:
        \[
        \begin{aligned}
            6^{1001} \mod 7 &= \left(6 \cdot 6^{1000}\right) \mod 7 \\
                            &= \left(6 \cdot 36^{500}\right) \mod 7 \\
                            &= (6 \mod 7) \cdot (36 \mod 7)^{500} \\
                            &= 6 \cdot 1 = 6
        \end{aligned}
        \]
\end{itemize}
    
\end{hwproblem}

\begin{hwproblem}
{34}{
    The Fibonacci numbers are $1,1,2,3,5,8,13,21,34, \ldots$. In general, the Fibonacci numbers are defined by $f_1=1, f_2=1$, and for $n \geq 3, f_n=f_{n-1}+f_{n-2}$. Prove that the $n$-th Fibonacci number $f_n$ satisfies $f_n<2^n$.
}
    
We will prove the claim of the problem using the Principal of Mathematical Induction II or strong induction. First, let \(T(n)\) be the statement ``The \(n\)-th Fibonacci number \(f_n \leq 2^n\).'' We will now show that the base case \(T(1)\) holds. Since \(f_1\) is defined to be \(1\), we have that
\[ f_1 = 1 \leq 2^1 = 2 \]
Thus, \(T(1)\) holds.

Now, let us prove the inductive case. Assume that \(T(1), T(2), \ldots, T(k)\) holds for some \(k \in \mathbb{Z}^+\). We will show that \(T(k+1)\) is true as well. That is, we will show that if \(f_1 \leq 2^1, f_2 \leq 2^2, \ldots, f_k \leq 2^k\), then \(f_{k+1} \leq 2^{k+1}\). Consider
\[
\begin{aligned}
    f_{k+1} &= f_{k} + f_{k-1} \\
            &\leq 2^k + 2^{k-1} \\
            &< 2^k + 2^k \\
            &= 2 \cdot 2^{k} = 2^{k+1}
\end{aligned}
\]
Thus, the inductive case holds as well.

Therefore, we can conclude by the Principal of Mathematical Induction II that the \(n\)-th Fibonacci number \(f_n \leq 2^n\)
\end{hwproblem}

\begin{hwproblem}
{42}{
    Suppose that a money order identification number and check digit of 21720421168 is erroneously copied as 27750421168. Will the check digit detect the error?
}
    
Recall that the check digit of a money order identification number is the modulo 9 of the first 10 digits. Notice that \(2775042116 \mod 9 = 8\) as well. Thus, the check digit will unfortunately not detect the error.
\end{hwproblem}

\begin{hwproblem}
{57}{
    Complete of the Theorem 0.8.
}
    
\begin{theorem}[Properties of Functions]
    Given functions \(\alpha: A \to B\), \(\beta: B \to C\), and \(\gamma: C \to D\), then
    \begin{enumerate}
        \item $\gamma(\beta \alpha)=(\gamma \beta) \alpha$ (associativity).
        \item If $\alpha$ and $\beta$ are one-to-one, then $\beta \alpha$ is one-to-one.
        \item If $\alpha$ and $\beta$ are onto, then $\beta \alpha$ is onto.
        \item If $\alpha$ is one-to-one and onto, then there is a function $\alpha^{-1}$ from $B$ onto $A$ such that $\left(\alpha^{-1} \alpha\right)(a)=a$ for all $a$ in $A$ and $\left(\alpha \alpha^{-1}\right)(b)=b$ for all $b$ in $B$.
    \end{enumerate}
\end{theorem}

\begin{proof}
    \phantom{pain}

    \underline{\textit{Part 1.}} Already proven by the textbook.

    \underline{\textit{Part 2.}} We will show that for every \(a_1, a_2 \in A\), if \(\beta\alpha(a_1) = \beta\alpha(a_2)\), then \(a_1 = a_2\).

    Let \(b_1 = \alpha(a_1)\) and \(b_2 = \alpha(a_2)\). Since \(\alpha\) is one-to-one, \(b_1\) must be the unique image of \(a_1\) under \(\alpha\). Similarly, \(b_2\) must be the unique image of \(a_2\) under \(\alpha\). Now we have that
    \[
        \beta\alpha(a_1) = \beta(\alpha(a_1)) = \beta(b_1) = \beta(b_2) = \beta(\alpha(a_2)) = \beta\alpha(a_2)
    \]
    Since \(\beta\) is one-to-one, we have that \(b_1 = b_2\). Consequently, we also have that \(a_1 = a_2\) since \(b_1\) and \(b_2\) are the unique images of \(a_1\) and \(a_2\) respectively.

    Therefore, we can conclude that if \(\alpha\) and \(\beta\) are one-to-one, then \(\beta\alpha\) is one-to-one.

    \underline{\textit{Part 3.}} We will show that for every element \(c \in C\), there exists at least one \(a \in A\) such that \(\beta\alpha(a) = c\).

    Let \(c \in C\) be given. Since \(\beta\) is onto, we have that there is at least one \(b \in B\) such that \(\beta(b) = c\). Similarly, since \(\alpha\) is onto, we have that there is at least one \(a \in A\) such that \(\alpha(a) = b\) for every \(b \in B\). 
    Thus, we can pick \(b_0\) such that \(\beta(b_0) = c\) and \(a_0\) such that \(\alpha(a_0) = b_0\). Finally, we have
    \[
    \begin{aligned}
        c &= \beta(b_0) \\
          &= \beta(\alpha(a_0)) \\
          &= \beta\alpha(a_0)
    \end{aligned}
    \]
    That is, for each \(c \in C\), there always exists an element \(a \in A\) such that \(\beta\alpha(a) = c\). Therefore, we can conclude that if \(\alpha\) and \(\beta\) are onto, then \(\beta\alpha\) is onto as well.

    \underline{\textit{Part 4.}} We will show that there exists an onto function \(\alpha^{-1}: B \to A\) such that it satisfies the following conditions.
    \begin{enumerate}
        \item For all \(a \in A\), \((\alpha^{-1}\alpha)(a) = a\).
        \item For all \(b \in B\), \((\alpha\alpha^{-1})(b) = b\).
    \end{enumerate}

    First, define \(\alpha^{-1}: B \to A\) to be the mapping such that for any \(b \in B\), \(\alpha^{-1}\) satisfies
    \[
        \alpha^{-1}(b) = a \iff \alpha(a) = b
    \]
    We now only need to show that this function exists, is onto, and satisfies the above conditions.

    \textit{Existence and Onto.} We will show that \(\alpha^{-1}\) is a valid function and that every \(a \in A\) is mapped to by some \(b \in B\). Since \(\alpha\) is one-to-one, we have that that for every \(a \in A\), \(\alpha(a)\) maps to a unique \(b \in B\). Since \(\alpha\) is also onto, we have that every \(b \in B\) is mapped to by some \(a \in A\).

    Consequently, we have that for every \(b \in B\), there is a unique \(a \in A\) that satisfies the condition \(\alpha(a) = b\), making \(\alpha^{-1}\) a valid function. That is, for each \(b \in B\), there is only a single output \(\alpha^{-1}(b) = a \in A\). Equivalently, since every \(a \in A\) satisfies \(\alpha(a) = b\) for some (unique) \(b \in B\), we have that every \(a \in A\) is mapped to by some \(b \in B\). Corollarily, \(\alpha^{-1}\) one-to-one as well.
    
    \textit{Condition 1.} Let \(a \in A\) be given and let \(b \in B\) be the unique image of \(a\) under \(\alpha\). That is, let \(b = \alpha(a)\). Observe that
    \[
    \begin{aligned}
        (\alpha^{-1}\alpha)(a) &= \alpha^{-1}(\alpha(a)) \\
                               &= \alpha^{-1}(b)
    \end{aligned}
    \]
    Let \(a' = \alpha^{-1}(b)\). This means that \(a'\) must satisfies \(\alpha(a') = b\). Since \(\alpha\) is one-to-one, we have that \(a' = a\). Thus, we have that \( (\alpha^{-1}\alpha)(a) = a \).

    \textit{Condition 2.} Let \(b \in B\) be given and let \(a \in A\) be the image of \(b\) under \(\alpha^{-1}\). That is, let \(a = \alpha^{-1}(b)\). This means that \(a\) must satisfy \(\alpha(a) = b\). Consider
    \[
    \begin{aligned}
        (\alpha\alpha^{-1})(b) &= \alpha(\alpha^{-1}(b)) \\
                               &= \alpha(a) \\
                               &= b
    \end{aligned}
    \]
    Thus, we have that \( (\alpha\alpha^{-1})(b) = b \).

    Therefore, we can conclude that there exists an onto function \(\alpha^{-1}: B \to A\) such that \((\alpha^{-1}\alpha)(a) = a\) and \((\alpha\alpha^{-1})(b) = b\).
\end{proof}
\end{hwproblem}
